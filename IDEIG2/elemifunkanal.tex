%\setcounter{chapter}{0}
%\bigskip

\section{Line�ris oper�torok}

Ha p�ld�ul $P:\X\rightarrow\X$ projekci�, azaz $P^2=P$, akkor a $Q:=I-P$ lek�lpez�s is projekci� (v�. pl. \cite{Kovacs}), tov�bb�
$$
\mathcal{N}(P)+\mathcal{R}(P)=\X\qquad\textrm{�s}\qquad\mathcal{N}(P)\cap\mathcal{R}(P)=\{0\},
$$
azaz $$\X=\mathcal{N}(P)\oplus\mathcal{R}(P).$$

\section{Tools from functional anmalysis}

For the sake of completeness I will remond you about some notations and fundamental theorems.
\begin{itemize}
  \item Let $\X^*:=L(\X,\R)$ denotes the \textbf{dual space} of $\X$. We introduce the duality product
      $$
      \langle f,x\rangle:=\langle f|x\rangle:=f(x)\qquad(x\in\X,\,f\in\X^*).
      $$
      For a subspace $M\subset\X$ and $N\subset\X^*$ we define
      $$
      M^\perp:=\left\{f\in\X^*:\;x\in\Y\;\Rightarrow\;\langle f,x\rangle=0\right\}\subset\X^*
      $$
      (the \textbf{annihilator} of $M$ or the \textbf{pseudo-orthogonal complement} to $M$)
      and
      $$
      N_\perp:=\left\{x\in\X:\;f\in N^*\;\Rightarrow\;\langle f,x\rangle=0\right\}\subset\X.
      $$
  \item (\textbf{Hahn-Banach.}) Let $\Y\subset\X$ be a linear subspace. If $\phi\in \Y^*$ then there is a $\Phi\in\X^*$ with
      $$
      \Phi(y)=\phi(y)\quad(y\in\Y)\qquad\textrm{and}\qquad\|\Phi\|_{\X^*}=\|\phi\|_{\X^*}.
      $$
      Corollaries.
      \begin{enumerate}
        \item For each nonzero vektor $u\in\X$ there is a continuous functional $f\in\X^*$ with
            $$
            \|f\|=1\qquad\textrm{and}\qquad f(u)=\|u\|.
            $$
        \item For each finite linear independent system $\{u_1,\ldots,u_d\}\subset\X$ there exist functionals  $\{f_1,\ldots,f_d\}\subset\X^*$ so we have a biorthogonal system $\{f_i,u_i\}$ $(i\in\{1,\ldots,d\})$:
            $$
            \langle f_i,u_j\rangle=\delta_{ij}\qquad(i,j\in\{1,\ldots,d\}).
            $$
      \end{enumerate}
  \item \textbf{Dual operator} of $A\in L(\X,\mathcal{Z})$ is defined by
      $$
      A^*f:=f\circ A\qquad(f\in\Y^*)
      $$
      for which
      $$
      A^*\in L(\mathcal{Z}^*,\X^*),\qquad\textrm{and}\qquad \|A^*\|_{L(\mathcal{Z}^*,\X^*)}=\|A\|_{L(\X,\mathcal{Z})}
      $$
      holds. Thus,
      $$
      \red{\boxed{\langle f,Ax\rangle}}=f(Ax)=(A^*(f))(x)=\red{\boxed{\langle A^*f,x\rangle}}\qquad(x\in\X,\,f\in\mathcal{Z}^*).
      $$
      For Hilbert spaces $(\X,\|\cdot\|_{\X})$ and $(\mathcal{Z},\|\cdot\|_{\mathcal{Z}})$ the $\X$ and $\mathcal{Z}$ can be identified with their dual and  $\langle\cdot,\cdot\rangle$ is the scalar product.
  \item \textbf{Closed-Range-Theorem.} Let $A\in L(\X,\mathcal{Z})$. Then the following conditions are equivalent:
      \begin{enumerate}
        \item $\mathcal{R}(A)$ is a closed subspace of $\mathcal{Z}$;
        \item $\mathcal{R}(A^*)$ is a closed subspace of $\mathcal{X}^*$
        \item $\mathcal{R}(A)=\mathcal{N}(A^*)_\perp\quad\Longleftrightarrow\quad\mathcal{R}(A^*)=\mathcal{N}(A)^\perp$ $\quad$ (\textbf{Fredholm alternative} means  for the equations
            $$
            (I)\;Ax=b\quad(x\in\X,\;b\in\mathcal{Z})\qquad\textrm{and}\qquad(II)\; A^*u=v\qquad(u\in\mathcal{Z}^*,v\in\X^*)
            $$
            that
            \begin{itemize}
              \item for the given $b\in\mathcal{Z}$ (I) has a solution iff for all solution $u$ of (II) $\langle u,b\rangle=0$ holds.
              \item for the given $v\in\X^*$ (II) has a solution iff for all solution $x$ of (I) $\langle v,x\rangle=0$ holds.
            \end{itemize}
            )
      \end{enumerate}
  \item Closed-Graph-Theorem. The linear operator $A:\X\rightarrow\mathcal{Z}$ is continuous iff $\left\{(x,z):\;Ax=z\right\}$ is closed in $\X\times \mathcal{Z}$ w.r.t. product topology $$\|(x,z)\|:\equiv\|x\|+\|z\|.$$
  \item \textbf{Projectors}.  Let $\X$ be a linear space. The linear operator $A:\X\rightarrow\mathcal{X}$ is to be called projector if it is idempotent: $A^2=A$.
      \begin{enumerate}
          \item $\mathcal{N}(I-A)=\mathcal{R}(A)$ and $\mathcal{R}(I-A)=\mathcal{N}(A)$.
          \item $\mathcal{X}=\mathcal{N}(A)\oplus \mathcal{R}(A)$, i.e. for $x\in \mathcal{X}$ we have
              $$
              x=u+v,\;u\in\mathcal{N}(A),\;v\in\mathcal{R}(A)\qquad\Longleftrightarrow\qquad u=(I-A)x,\;v=Ax.
              $$
          \item The operator $I-A$ is also a projector.
          \item If $\X=U\oplus V$ then for a projector $A:\X\rightarrow U$ along $V$ the following statement is true:
              $$
              A\in L(\X)\qquad\Longleftrightarrow\qquad (U,V\textrm{ are closed in }\X).
              $$
      \end{enumerate}
\end{itemize} 
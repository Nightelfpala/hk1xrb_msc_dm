\documentclass[12pt,a4paper,oneside]{book}
\usepackage[latin2]{inputenc}
\linespread{1.2}
\usepackage[latin2]{inputenc}
\usepackage{t1enc}
\def\magyarOptions{defaults=hu-min}
\usepackage[magyar]{babel}
\usepackage{watermark}
\usepackage{funk1}
\usepackage{makeidx}
\usepackage{fancyhdr}
\usepackage{graphicx}
\usepackage{graphics}
\usepackage{indentfirst}
\usepackage{footnote}
\usepackage{xcolor}
%%%%%%%%%%%%%%%%%%%%%%%%%%%%%%%%%%%%
%Alapértelmezések
%%%%%%%%%%%%%%%%%%%%%%%%%%%%%%%%%%%
\setlength{\topmargin}{-1cm} \setlength{\oddsidemargin}{.1cm}
\setlength{\evensidemargin}{\oddsidemargin}
\setlength{\textwidth}{16cm} \setlength{\textheight}{24cm}
\setlength{\headheight}{1cm} \setlength{\headsep}{4mm}
%%%%%%%%%%%%%%%%%%%%%%%%%%%%%%%%%%%%%

\usepackage{amssymb,amsmath,amsfonts,amssymb,amsbsy}

\usepackage{url}
\usepackage{appendix}
\usepackage{funk4}

\makeindex

%% SZOVEGKIEMELESEK
%%
%%%%%%%%%%%%%%%%%%%%%%%\setlength\mathindent{0 mm}
\newcommand{\ki}[1]{\textbf{\textsl{#1}}}
\newcommand{\key}[1]{\textbf{#1}}
\def\idez#1{{\ideze}#1''}
\def\ideze{\setbox0=\hbox{\lower1.38ex\hbox{''}}\dp0=0pt\box0}
%%
%%
\newcommand*{\bigtimes}{\mathop{\raisebox{-.5ex}{\hbox{\huge{$\times$}}}}}
%%
%\makeindex
%\newindex{index2}{raw2}{comp2}{Névmutató}
%\usepackage{funk0}

%\newindex{aut}{adx}{and}{Névmutató}


\usepackage[normalem]{ulem}

\newcommand{\green}[1]{{\color{green}{ #1}}}
\newcommand{\red}[1]{{\color{red}{ #1}}}
\newcommand{\blue}[1]{{\color{blue}{ #1}}}
\newcommand{\yellow}[1]{{\color{yellow}{ #1}}}
%\newcommand{\yellow}[1]{{\color{yellow}{ #1}}}
\newcommand{\violet}[1]{{\color{violet}{ #1}}}

\hyphenation{fej-l�-d�s-bi-o-l�-gi-a-i rend-szer sza-ka-szon-k�nt de-r�k-sz�-g�
	meg-ol-d�s meny-nyi-s�-ge meg-ol-d�-s�-nak}

\begin{document}
	


%=============================================================

\definecolor{magenta}{rgb}{1,0.5,0.14}



\input titelseite.tex
\input Vorwort.tex
\tableofcontents

%\chapter{Jel�l�seksek}
%\input bezeichnungen.tex

\chapter{Funkcion�lanalitikai seg�deszk�z�k}
\input faktorter.tex
\input operator.tex
\input kompaktoperator.tex
\input fredholmoperator.tex
\input implicit.tex
\chapter{Bifurk�ci�k}
\input Bifurkationen.tex
\chapter{A Ljapunov-Schmidt-redukci�}
\input LjapunovSchmidt.tex

\chapter{Alkalmaz�sok}
Az al�bbiakban a Ljapunov-Schmidt-m�dszert fogjuk haszn�lni perem�rt�k-feladatok megoldhat�s�g�nak vizsg�lat�ra.
\input Alkalmaz1.tex
\input Alkalmaz2.tex

\chapter{A numerikus Ljapunov-Schmidt-m�dszer}
\input NumLSMethod.tex



\input hivatkozasok.tex




\clearpage
\thispagestyle{empty}
\begin{flushleft}
\textbf{\hypertarget{bliptak}{Lipt�k Bence,\\ programtervez� informatikus szakos egyetemi hallgat�\\
E�tv�s Lor�nd Tudom�nyegyetem, Informatikai Kar,\\
1117 Budapest, P�zm�ny P�ter s�t�ny 1/C,\\
padsoldier@gmail.com\\
%\url{http://numanal.inf.elte.hu/~alex}
}}
\end{flushleft}


\end{document} 
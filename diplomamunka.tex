\documentclass[oneside, titlepage, 12pt, a4paper]{report}

\usepackage{amsmath}	% align
\usepackage{amsfonts}	% \mathbb
\usepackage{amssymb}	% \mathcal
\usepackage{mathtools}	% \norm definiálás
\usepackage[magyar]{babel}	% tartalomjegyzék felirat
\usepackage[utf8]{inputenc}	% magyar karakterek
\usepackage[T1]{fontenc}
\usepackage{graphicx}	% ábrák
\usepackage{setspace}	% sorköz \onehalfspacing

\newtheorem{theorem}{Tétel}[section]
\newtheorem{lemma}{Lemma}[section]
\newtheorem{definition}{Definíció}[section]
\newtheorem{statement}{Állítás}[section]

\DeclareMathOperator{\Ima}{Im}	% képtér operátor
\DeclareMathOperator{\Ker}{Ker}	% magtér operátor
\DeclareMathOperator{\coker}{coker}	% magtér operátor
\DeclareMathOperator{\codim}{codim}	% kodimenzió operátor
\DeclareMathOperator{\ind}{ind}	% index operátor
\DeclareMathOperator{\Span}{span}	% magtér operátor

\DeclarePairedDelimiter\norm{\lVert}{\rVert}	% norma

\textwidth=6.truein \textheight=9.truein
\hoffset=-.5truein
\voffset=-.8truein

\begin{document}
\begin{titlepage}
% belső fedőlap

% minta forrása: https://github.com/shdnx/ELTE-LaTeX-Thesis-Base
\begin{minipage}{0.40\linewidth}
\includegraphics[scale=0.8]{./abrak/elte_logo_szines.jpg}
\end{minipage}
\begin{minipage}{0.50\linewidth}
\begin{center}
Eötvös Loránd Tudományegyetem \\
Informatikai Kar \\
Numerikus Analízis Tanszék
\end{center}
\end{minipage}

\hrule
\vfill

\begin{center}
\Huge
\textbf{A Ljapunov-Schmidt-módszer}
\normalsize
\end{center}

\vfill

\begin{minipage}[t]{0.5\linewidth}
\begin{flushleft}
\textbf{Dr. Kovács Sándor} \\
Adjunktus
\end{flushleft}
\end{minipage}
\begin{minipage}[t]{0.5\linewidth}
\begin{flushright}
\textbf{Lipták Bence Gábor} \\
Programtervező Informatikus MSc
\end{flushright}
\end{minipage}

\vfill

\begin{center}
Budapest, 2018.
\end{center}

\end{titlepage}

\tableofcontents

%% Bevezetés

\onehalfspacing
\chapter{Bevezetés}
\label{chap:Introduction}

% TODO bevezetés szövege

%% Alapozás: faktorterek, Fredholm operátorok

\onehalfspacing
\chapter{Funkcionálanalízis kiegészítés}	% TODOtalán cím
\label{chap:Funcanal_ext}

Ahhoz, hogy a Ljapunov-Schmidt-módszert ismertethessük szükségünk van a Fredholm-operátorok fogalmára, amihez elengedhetetlenek a faktorterek és a kompakt operátorok. A módszerhez emellett még az implicit függvény tétel (Banach-terekben) is szükséges.

% TODOtalán jelölések szekció: Lin op, korl Lin op halmazai

%
\section{Faktorterek}
\label{sec:Quotient_space}

Először is ismertessük a faktorterek definícióját, és az alkalmazásunk szempontjából fontos tulajdonságait, a \cite{faktorter} könyv 4.2 fejezete alapján, ahol a bizonyítások is megtalálhatóak.
\begin{definition}[Faktortér]
Legyen V egy $\mathbb{K}$ feletti lineáris tér, $U \subset V$ pedig egy altere. A V tér U szerinti \textbf{faktortere} vagy \textbf{hányadostere}
\begin{equation}
V / U := \{v + U \mid v \in V\},
\end{equation}
ahol
\begin{equation}
v + U := \{v + u \mid u \in U \}.
\end{equation}
\end{definition}

Így egy lineáris térben egy altér segítségével definiáltunk egy halmazrendszert. A következő állítással megfogalmazzuk, hogy a kapott halmazok és az U altér között mi az összefüggés.
\begin{statement}
Ha V egy $\mathbb{K}$ feletti lineáris tér, $U \subset V$ egy altere, akkor $\forall v, v' \in V$-re
\begin{equation}
v + U = v' + U \qquad \Leftrightarrow \qquad v - v' \in U.
\end{equation}
\end{statement}

Ennek segítségével belátható, hogy ha  ''$v - v' \in U$'' feltétellel definiálunk egy relációt $V$ elemein, akkor az egy ekvivalenciareláció és az ekvivalenciaosztályok pedig a $V / U$ faktortér elemei. Ezután definiáljunk műveleteket a faktortéren.
\begin{theorem}
Legyen V egy $\mathbb{K}$ feletti lineáris tér, $U \subset V$ egy altere, ekkor
\begin{itemize}
\item
Megadunk
\begin{align*}
V / U \times V / U & \longrightarrow V / U \\
\mathbb{K} \times V / U & \longrightarrow V / U
\end{align*}
leképezéseket a következő módon
\begin{align}
(v_1 + U) + (v_2 + U) &:= (v_1 + v_2) + U \\
\alpha (v + U) &:= \alpha v + U
\end{align}
melyek jól definiáltak.
\item
A $V / U$ faktortér ezekkel a műveletekkel egy $\mathbb{K}$ feletti lineáris tér.
\item
$\pi_U$ az úgynevezett \textbf{kanonikus leképezés}, melyet
\begin{equation*}
\pi_U(v) := v + U \qquad (v \in V)
\end{equation*}
módon definiálunk lineáris operátor $V$ és $V / U$ között.
\end{itemize}
\end{theorem}

A faktortér konstrukciója nagyon hasonlít az egész számok körében létesített maradékosztályokra, és rögzített $U \subset V$ esetén ugyanaz a jelölés is alkalmazható:
\begin{equation*}
\overline{v} := v + U \qquad (v \in V),
\end{equation*}
ezzel a jelöléssel a műveletek:
\begin{align*}
\overline{v} + \overline{w} &= \overline{v + w} &(v, w \in V) \\
\alpha \overline{v} &= \overline{\alpha v} &(\alpha \in \mathbb{K}, v \in V).
\end{align*}

Még fontos észrevétel, hogy a $\pi_U$ leképezés magtere pontosan az $U$ halmaz, valamint az operátor szürjektív is, amiből kapunk a faktortér dimenziójára egy összefüggést:
\begin{theorem}
Legyen V egy $\mathbb{K}$ feletti lineáris tér, $U \subset V$ egy altere, ekkor
\begin{equation}
\dim V / U + \dim U = \dim V.
\end{equation}
\end{theorem}

Ha egy $v \in V$ elemet egy $u \in U$ elemmel eltolunk ($U \subset V$ altér), akkor a $V / U$ faktortérbeli $\pi_U$ általi képe változatlan, ez a konstrukció alkalmas arra, hogy olyan függvényeket vizsgáljunk, amiknek az értéke az U altéren konstans, speciális esetben 0. Az ilyen leképezésekről szól a következő tétel:	% TODOtalán átfogalmazni
\begin{theorem}[Homomorfiatétel vektorterekre]
Legyen V egy $\mathbb{K}$ feletti lineáris tér, $F : V \rightarrow W$ egy lineáris leképezés, $U \subset V$ egy altér amire $U \subset \Ker F$. Ekkor egyértelműen létezik $F' : V / U \rightarrow W$ amivel $F = F' \circ \pi_U$. Emellett
\begin{itemize}
\item
$\Ima F = \Ima F'$, illetve F' pontosan akkor szürjektív, amikor F is,
\item
$\Ker F' = (\Ker F) / U$, illetve F' pontosan akkor injektív, amikor $U = \Ker F$.
\end{itemize}
F'-t az \textbf{F által indukált homomorfizmusnak} nevezzük.	% TODOmagyar név helyes-e
\end{theorem}

A faktortérnek van egy hasznos tulajdonsága, amivel nem halmazrendszerként, hanem altérként lehet kezelni.
\begin{theorem}
Legyen V egy $\mathbb{K}$ feletti lineáris tér, $U \subset V$ egy altere, $W \subset V$ pedig az U komplemens altere (tehát $V = W \oplus U$). Ekkor a $\pi_U$ kanonikus leképezés leszűkítése W-re
\begin{equation*}
\pi_{\mkern 1mu \vrule height 2ex\mkern2mu U} : W \longrightarrow V / U, \; \pi_{\mkern 1mu \vrule height 2ex\mkern2mu U}(w) = w + U
\end{equation*}
izomorfia, azaz $W \cong V / U$.
\end{theorem}

% TODOtalán kivezető szöveg

% TODOtalán 4.2.8 tétel ~ ennek analógja funkcionálokra (ha ezt később használjuk)

%
\section{Kompakt operátorok}
\label{sec:kompakt}

A Fredholm-operátorok konstrukciójához érdemes feleleveníteni a kompakt operátorok fogalmát és néhány, a gyakorlat szempontjából hasznos tulajdonságukat. A következők során $(X, \norm{.}_X)$ és $(Y, \norm{.}_Y)$ normált terek.
\begin{definition}[Kompakt operátor]
$A : X \rightarrow Y$ operátor \textbf{kompakt}, ha bármely $U \subset X$ korlátos halmaznak a képe prekompakt, azaz $\overline{A[U]} \subset Y$ kompakt, valamint ha A korlátos, akkor \textbf{teljesen folytonosnak} is nevezzük. \cite{funkanal}
\end{definition}

A továbbiakban az $X \rightarrow Y$ közötti kompakt és lineáris operátorok halmazát $K(X, Y)$-al, $X = Y$ esetén $K(X)$-szel jelöljük, ez zárt alteret alkot $L(X, Y)$-ban.
\begin{definition}
$A \in L(X, Y)$ operátor \textbf{véges rangú}, ha a képtere véges dimenziós ($\dim \Ima A < \infty$). A véges rangú operátorok halmazát $K_{fin}(X, Y)$-al rövidítjük.
\cite{funkanal,FcNotex}
\end{definition}
Belátható (\cite{funkanal}), hogy minden véges rangú operátor kompakt (és így a jelölés indokolt). \par

A kompakt operátorok bizonyos esetekben közelíthetőek véges rangú operátorokkal:
\begin{theorem}
Ha Y-ban van Schauder-bázis, akkor $A \in L(X, Y)$ pontosan akkor kompakt, ha határértéke véges rangú operátorok valamely sorozatának, azaz $K(X, Y) = \overline{K_{fin}(X, Y)}$. \cite{FcNotex}
\end{theorem}
Ezek a feltételek teljesülnek például szeparábilis Hilbert-terekben vagy $L^p$-terekben ($p \ge 1$). \par

Még tegyünk egy megállapítást a kompakt operátorok adjungáltjával kapcsolatban:
\begin{theorem}
Legyenek $(X, \norm{.}_X)$ és $(Y, \norm{.}_Y)$ Banach-terek, $A \in L(X, Y)$, ekkor
\begin{equation*}
A \in K(X, Y) \Leftrightarrow A^* \in K(Y^*, X^*),
\end{equation*}
azaz A pontosan akkor kompakt, ha $A^*$ adjungált operátora kompakt.
\cite{funkanal, lectures16and17}
\end{theorem}
% TODOtalán kivezető szöveg

%
\section{Fredholm-operátorok}
\label{sec:Fredholm}

Végül beszéljünk a Fredholm-operárokról és lehetőségekről az előállításukra. A továbbiakban $(X, \norm{.}_X)$ és $(Y, \norm{.}_Y)$ Banach-terek.
\begin{definition}[Fredholm-operátor]
$T \in L(X, Y)$ \textbf{Fredholm-operátor}, ha az alábbiak teljesülnek:
\begin{itemize}
\item
$\dim \Ker T < \infty$,
\item
T[X] zárt Y-ban,
\item
$\dim (Y / T[X]) < \infty$.
\end{itemize} % TODOmagyar nevek
Ekkor a T operátor \textbf{indexe} $\ind T := \dim \Ker T - \dim (Y / T[X]) \in \mathbb{Z}$, T \textbf{cokernele} $\coker T := Y / T[X]$ (azaz Y-ban a T képtere szerinti faktortér), valamint a cokernel dimenziója az operátor \textbf{kodimenziója} $\codim T := \dim \coker T$. \cite{faktorter, FcNotex, lectures16and17, diffun2}
\end{definition}
Az X és Y közötti Fredholm-operátorok halmazát a továbbiakban $\mathcal{F}(X, Y)$-al jelöljük. Belátható, hogy a második feltétel (T képterének a zártsága) a másik kettőből következik \cite{lectures16and17, diffun2}. \par

Most nézzük meg, hogy bizonyos függvény műveletek hatására változik-e a Fredholm-tulajdonság.	% TODOtalán átírni
\begin{theorem}
Legyenek $(X, \norm{.}_X)$, $(Y, \norm{.}_Y)$ és $(Z, \norm{.}_Z)$ Banach-terek, ekkor:
\begin{itemize}
\item
$A \in \mathcal{F}(X, Y)$ és $B \in \mathcal{F}(Y, Z)$, ekkor $B \circ A \in \mathcal{F}(X, Z)$ és $ \ind (B \circ A) = \ind B + \ind A$,
\item
$A \in \mathcal{F}(X, Y)$, ekkor $A^* \in \mathcal{F}(X^*, Y^*)$ és $\ind A^* = - \ind A$,
\item
$\mathcal{F}(X, Y)$ nyílt részhalmaza L(X, Y)-nak, és az $\ind : \mathcal{F}(X, Y) \rightarrow \mathbb{Z}$ függvény lokálisan konstans.
\end{itemize}
Tehát Fredholm-operátorok kompozíciója Fredholm-operátor, valamint Fredholm-operátor adjungáltja is Fredholm-operátor. \cite{FcNotex, diffun2}
\end{theorem}
% TODOtalán a lokálisan konstansságból következően a kis perturbáció tétel is?

Korábban említettük a kompakt operátorok és a Fredholm-operátorok kapcsolatát, erről szól a következő állítás.
\begin{theorem}
$T \in L(X, Y)$ bijektív, $K \in L(X, Y)$ kompakt, ekkor $T + K$ Fredholm-operátor és $\ind(T + K) = 0$. Ennek speciális esete amikor $(X, \norm{.}_X) = (Y, \norm{.}_Y)$ és T az identitás X-en. \cite{FcNotex,diffun2}
\end{theorem}

Amennyiben ilyen módon kaptunk egy Fredholm-operátort, akkor ahhoz (a kompakt operátorok altér-tulajdonságából kifolyólag) egy kompakt operátort hozzáadva is Fredholm-operátort kapunk, ez igaz tetszőleges konstrukció esetén is:
\begin{theorem}
$K \in K(X, Y)$ és $F \in \mathcal{F}(X, Y)$ esetén $K + F \in \mathcal{F}(X, Y)$, valamint $\ind(K + F) = \ind F$. \cite{FcNotex}
\end{theorem}

Mivel a Fredholm-operátoroknál nem feltétel, hogy a magterük csak a tér nullelemét tartalmazza, ezért általában nem invertálhatóak, de egy hasonló tulajdonságot megadhatunk:	% TODOtalán átírni
\begin{theorem}
$T \in L(X, Y)$ pontosan akkor Fredholm-operátor, ha létezik $B \in L(Y, X)$, $K_X \in K(X)$, $K_Y \in K(Y)$ úgy, hogy
\begin{equation*}
BT = I_{\mkern 1mu \vrule height 2ex\mkern2mu X} + K_X, TB = I_{\mkern 1mu \vrule height 2ex\mkern2mu Y} + K_Y.
\end{equation*}
Azaz a Fredholm-tulajdonság lényegében az invertálhatóság modulo kompakt operátort jelenti. \cite{FcNotex, diffun2}
\end{theorem}
% TODOtalán kivezető szöveg

%
\section{Implicit függvény tétel}
\label{sec:implicitfvtetel}

A módszer használata során az eredmény eléréséhez szükségünk lesz az implicit függvény tételre Banach-terekben, illetve ahhoz kötődően a Fréchet-derivált fogalmára. $(X, \norm{.}_X)$, $(Y, \norm{.}_Y)$ és $(Z, \norm{.}_Z)$ a következők során Banach-terek.

\begin{definition}
$F : X \times Y \rightarrow Z$ \textbf{Fréchet-differenciálható} X-ben az $(u_0, v_0)$ pontban, ha létezik $(D_xF)(u_0, v_0) \in L(X, Z)$ úgy, hogy
\begin{equation*}
\lim_{h \to 0} \frac{\norm{F(u_0 + h, v_0) - F(u_0, v_0) - (D_xF)(u_0, v_0)h}_Z}{\norm{h}_X} = 0.
\end{equation*}
\end{definition}
Látható, hogy egy ponthoz tartozó derivált (a valós, többdimenziós esethez hasonlóan) egy függvény.

\begin{theorem}[Implicit függvény tétel Banach-terekben]
\label{implicit}
$F : X \times Y \rightarrow Z$ folytonos, $(u_0, v_0) \in X \times Y$, $F(u_0, v_0) = 0$, $(D_xF)(u_0, v_0)$ bijektív és folytonos. Ekkor létezik $(u_0, v_0)$-nak olyan $U \times V \subset X \times Y$ környezete és $G:V \rightarrow U$ függvény, amivel $G(v_0) = u_0$ és
\begin{equation*}
F(G(v), v) = 0 \qquad (\forall v \in V).
\end{equation*}
Ezen felül minden $U \times V$-beli megoldás ebben a formában áll elő. \cite{IFaLS}
\end{theorem}
Az ilyen tulajdonságokkal rendelkező $D_xF$ függvényeket lineáris homeomorfizmusnak nevezzük:
\begin{definition}
Az $A$ leképezés \textbf{lineáris homeomorfizmus}, ha folytonos, bijektív és az inverze is folytonos.
\end{definition}
Az inverz folytonossága pedig a Banach-féle inverz tételből (vagy Banach-féle homeomorfia-tételből) következik:
\begin{theorem}
$A \in L(X, Y)$ Banach-terek közötti bijektív operátor, ekkor $A^{-1} \in L(Y, X)$. (\cite{funkanal} 6.1.4) % TODO kell-e pontos helymegjelölés
\end{theorem}
% TODOtalán kivezető szöveg
% TODOtalán Taylor sorfejtés Banach-térben

%%

\onehalfspacing
\chapter{A Ljapunov-Schmidt-módszer és alkalmazásai}
\label{chap:LjapunovSchmidt}

\section{A Ljapunov-Schmidt-redukció}
\label{sec:LSMethod}
A Ljapunov-Schmidt-módszer célja, hogy egy végtelen-dimenziós feladatot egy jobban kezelhető, véges dimenziós feladatra redukáljunk. \par
Először is nézzük meg az általános esetet (\cite{8.6TLSM, chapter2chapter10}). $(X, \norm{.}_X)$, $(Y, \norm{.}_Y)$ és $(\Lambda, \norm{.}_\Lambda)$ legyenek Banach-terek ($\Lambda$ a paramétertér szerepét tölti be), 0-val pedig értelemszerűen a megfelelő tér nullelemét jelöljük.\par
Legyen $F : X \times \Lambda \rightarrow Z$ egy korlátos, $p$-szer folytonosan differenciálható operátor ($p \ge 1$) úgy, hogy
\begin{equation*}
F(x_0, \lambda_0) = 0,
\end{equation*}
ahol $(x_0, \lambda_0) \in X \times \Lambda$. Keressük az 
\begin{equation}
\label{eq:LSproblem}
F(x, \lambda) = 0
\end{equation}
egyenlet olyan megoldásait $(x_0, \lambda_0)$ környezetében, amikre $x \ne x_0$. \par
Először tegyük fel, hogy $D_xF(x_0, \lambda_0)$ izomorfizmus. Ekkor alkalmazható az implicit függvény tétel, azaz létezik $ U \times V \subset X \times \Lambda$ és $f : V \rightarrow U$, amivel
\begin{align*}
f(\lambda_0) &= x_0, \\
F(f(\lambda), \lambda) &= 0 \qquad(\lambda \in V),
\end{align*}
és $U \times V$-ben minden megoldás előáll ebben az alakban. \par
A Ljapunov-Schmidt-módszer bizonyos feltételek mellett megoldja a feladatot abban az esetben, ha $D_xF(x_0, \lambda_0)$ nem izomorfizmus, tételezzük tehát ezt fel. A továbbiakban a $D_xF(x_0, \lambda_0)$ operátort rövidítve, $D_xF$-el jelöljük (ami egy $L(X, Z)$ függvény). \par
Legyen $K := \Ker D_xF$ és $R := \Ima D_xF$, és tekintsük a hozzájuk tartozó kiegészítő altereket:
\begin{align*}
X &= K \oplus W, \\
Z &= R \oplus S.
\end{align*}
Tegyük fel, hogy $K$ és $S$ véges dimenziósak, azaz $D_xF$ Fredholm-operátor, ekkor létezik megfelelő $W$ és $S$, melyek zártak (\cite{funkanal} 4.5.9).	% TODO kell pontos hely megjelölés a hivatkozáshoz?
Tekintsük $D_xF_{\mkern 1mu \vrule height 2ex\mkern2mu W} : W \rightarrow R$ függvényt, ez bijekció és R zárt, ezért létezik $D_xF_{\mkern 1mu \vrule height 2ex\mkern2mu W}$-nek folytonos inverze. \par % TODOtalán indoklás: (szürjektivitás $R$ képtér tulajdonságából következik, injektivitás abból, hogy $\Ker D_xF_{\mkern 1mu \vrule height 2ex\mkern2mu W} = W \cap K = \{0\}$)
Bontsuk fel minden $x \in X$-et fel $K$ mentén egyértelműen ($x_0$ esetén külön jelölve a kapott részeket):
\begin{align*}
x &= x_1 + x_2 \qquad &(x_1 \in K, x_2 \in W), \\
x_0 &= x_1^{(0)} + x_2^{(0)} \qquad &(x_1^{(0)} \in K, x_2^{(0)} \in W),
\end{align*}
Legyen $P_R : Z \rightarrow R$ projekciós operátor $R$-re. Alkalmazzuk $P_R$-t és $(I - P_R)$-t  az eredeti (\ref{eq:LSproblem}) feladatra, kiegészítve az $x$ szétválasztásával:
\begin{align*}
P_RF(x_1, x_2, \lambda) & := P_RF(x_1 + x_2, \lambda) = 0, \\
(I - P_R)F(x_1, x_2, \lambda) & := (I - P_R)F(x_1 + x_2, \lambda) = 0,
\end{align*}
ahol így
\begin{align*}
P_RF &: K \times W \times \Lambda \rightarrow R, \\
(I - P_R)F &: K \times W \times \Lambda \rightarrow S,
\end{align*}
az elsőt kissé átrendezve
\begin{equation*}
P_RF : W \times (K \times \Lambda) \rightarrow R,
\end{equation*}
alakot kapunk, amire a $D_xF_{\mkern 1mu \vrule height 2ex\mkern2mu W}$ invertálhatósága (és az inverz folytonossága) lehetővé teszi az implicit függvény tétel alkalmazását, amivel kapunk egy $x_2 : \Theta \times \Gamma \rightarrow W$ függvényt (alkalmas $\Theta \subset K$ és $\Gamma \subset \Lambda$ halmazokkal $x_1^{(0)}$ és $\lambda_0$ környezetében), amivel
\begin{align*}
x_2 &:= x_2(x_1, \lambda) \qquad (x_1 \in \Theta, \lambda \in \Gamma), \\
x_2^{(0)} &= x_2(x_1^{(0)}, \lambda_0), \\
P_RF&(x_1 + x_2(x_1, \lambda), \lambda) = 0
\end{align*}
teljesül, így már csak
\begin{equation}
(I - P_R)F(x_1 + x_2(x_1, \lambda), \lambda) = 0
\end{equation}
egyenletet kell megoldani, ami így $\Theta \times \Gamma \subset K \times \Lambda$ halmazon értelmezett, $S$-beli értékeket vesz fel. \par
Ha még azt is feltesszük, hogy $\Lambda$ véges dimenziós, akkor $\dim K < \infty$ és $\dim S < \infty$ miatt kapunk egy véges sok egyenletből álló egyenletrendszert véges sok ismeretlennel, amit már például a ``szokásos'' implicit függvény tétellel megkísérelhetünk megoldani. \par
Az eredeti \ref{eq:LSproblem} feladat például akkor állhat elő, ha van $A : X \rightarrow Z$ korlátos lineáris operátorunk, $N : X \times Y \rightarrow Z$ folytonosan differenciálható leképezésünk (ami nem feltétlenül lineáris) és az
\begin{equation*}
Ax = N(x, \lambda)
\end{equation*}
egyenlet nemtriviális $x \in X$ megoldásait keressük (\cite{8.6TLSM}).

% TODO bifurkációs dolgok (lokális, Hopf) {8.6TLSM, chapter2chapter10}

% TODO bifurkációs pont definíció

\section{Hopf bifurkáció}
\label{sec:Hopfbifurcation}
A következőkben az úgynevezett Hopf-bifurkáció-tételt vizsgáljuk, ami időfüggetlen, egy paramétertől függő differenciálegyenlet-rendszerek nemtriviális periodikus megoldásainak a létezéséhez ad meg feltételeket. Ennek során fel fogjuk használni az előző fejezetben tárgyalt Ljapunov-Schmidt-módszert. \cite{chapter2chapter10} \par
Legyen $f : \mathbb{R}^n \times \mathbb{R} \rightarrow \mathbb{R}^n$ kétszer folytonosan differenciálható leképezés, melyre
\begin{equation*}
f(0, \alpha) = 0 \qquad (\alpha \in \mathbb{R}).
\end{equation*}
A vizsgálandó differenciálegyenlet-rendszer, melynek $u \in \mathbb{R} \rightarrow \mathbb{R}^n$ megoldásait keressük a következő:
\begin{equation}
\label{eq:HopfDE}
\frac{du}{dt} + f(u, \alpha) = 0.
\end{equation}
Tegyük fel, hogy $f$-re teljesülnek az alábbiak:
\begin{itemize}
\item
valamilyen $\alpha = \alpha_0$ érték esetén $i = \sqrt{-1}$ és $-i$ sajátértékei $D_uf(0, \alpha_0)$-nak, és $\pm ki$ ($k = 0, 2, 3, \dots$) nem sajátértékek;
\item
$\alpha_0$ egy környezetében van a sajátértékeknek és sajátvektoroknak egy-egy görbéje ($\beta$ illetve $a$), amivel
\begin{align}
D_uf(0, \alpha_0) a(\alpha) &= \beta(\alpha) a(\alpha), \\
a(\alpha_0) &\ne 0, \\
\beta(\alpha_0) &= i, \\
\Re(\frac{d\beta}{d\alpha}(\alpha_0)) &\ne 0.
\end{align}
\end{itemize}
A célunk az, hogy találjunk olyan $\eta > 0$ számot illetve egy
\begin{equation*}
(u, \rho, \alpha) : (-\eta, \eta) \rightarrow C_{2\pi}^1(\mathbb{R}, \mathbb{R}^n) \times \mathbb{R} \times \mathbb{R}
\end{equation*}
folytonosan differenciálható függvényt, amivel $(u(s), \rho(s), \alpha(s))$ megoldja a (\ref{eq:HopfDE}) egyenlet egy általánosítását:
\begin{equation}
\label{eq:HopfDErho}
\frac{du}{dt} + \rho f(u, \alpha) = 0
\end{equation}
nemtriviális módon ($u(0) \ne 0$ ha $s \ne 0$),
\begin{equation*}
\rho(0) = 1, \> \alpha(0) = \alpha_0, \> u(0) = 0,
\end{equation*}
és ezen felül minden megoldás $(0, 1, \alpha_0)$ közelében valamely $s \in (-\eta, \eta)$ esetén előálljon ($u$-beli fáziseltolástól eltekintve). \par
Vegyük észre, hogy ha $u(t)$ megoldása (\ref{eq:HopfDE})-nek $2 \pi \rho$ periódussal, akkor $u(\rho t)$ $2\pi$-periódusú megoldása (\ref{eq:HopfDErho})-nak.\par
Legyenek $X = C^1_{2\pi}(\mathbb{R}, \mathbb{R}^n)$ és $Y = C_{2\pi}(\mathbb{R}, \mathbb{R}^n)$ Banach-terek a ``szokásos'' normákkal. Definiáljunk egy $F : X \times \mathbb{R} \times \mathbb{R} \rightarrow Y$ operátort a következő módon:
\begin{equation*}
(u, \rho, \alpha) \mapsto \frac{du}{dt} + \rho f(u, \alpha),
\end{equation*}
ami így kétszer folytonosan differenciálható, és keressük az
\begin{equation}
\label{eq:HopfDEbigF}
F(u, \rho, \alpha) = 0
\end{equation}
megoldásait, amikor $\rho$ közel van $1$-hez, $\alpha$ közel van $\alpha_0$-hoz és $u \ne 0$. Látható, hogy
\begin{equation*}
F(0, \rho, \alpha) = 0 \qquad (\rho \in \mathbb{R}, \alpha \in \mathbb{R}).
\end{equation*}
Vegyük észre, hogy
\begin{align*}
\phi_0(t) &= \Re(e^{it}a(\alpha_0)), \\
\phi_1(t) &= \Im(e^{it}a(\alpha_0))
\end{align*}
függvények $2\pi$-periodikus megoldásai az
\begin{equation}
\frac{du}{dt} + D_uf(0, \alpha_0)u = 0
\end{equation}
egyenletnek, és ezek a függvények kifeszítik $D_uF(0, 1, \alpha_0)$ magterét:
\begin{equation*}
\Ker D_uF(0, 1, \alpha_0) = \Span \{ \phi_0, \phi_1 = \phi_0'\} \subset X.
\end{equation*}
% TODO lin DE theory hivatkozás
A differenciálegyenletek elméletéből következik, hogy $\Ima D_uF(0, 1, \alpha_0) \subset Y$ zárt és
\begin{equation*}
\Ima D_uF(0, 1, \alpha_0) = \{ g \in Y : \langle g, \psi_i \rangle = 0, i = 0, 1\},
\end{equation*}
ahol $\langle \cdot , \cdot \rangle$ az $L^2$-beli skaláris szorzás, $\{\psi_0, \psi_1 \}$ pedig $\Ker \{ -u' + D_uf^T(0, \alpha_0)u\}$ bázisa, ami az $u' + D_uf(0, \alpha_0)u$ adjungált operátora. Emellett $\psi_1 = \psi_0'$ és $\langle \phi_i, \psi_j \rangle = \delta_{ij}$ Kronecker-deltával egyenlő. Ebből következik, hogy $D_uF : X \rightarrow Y$ Fredholm-operátor aminek a magtere és a kokernele is 2 dimenziójú. \par
Így alkalmazhatjuk a Ljapunov-Schmidt-módszert:
\begin{align*}
X = V \oplus W, \\
Y = Z \oplus T,
\end{align*}
ahol $V = \Ker D_uF$ és $T = \Ima D_uF$. $U$ legyen a $(0, 1, \alpha_0, 0) \in V \times \mathbb{R} \times \mathbb{R} \times \mathbb{R}$, és $G$ legyen $U$-n értelmezve a következő módon:
\begin{equation*}
G(v, \rho, \alpha, s) =
\begin{cases}
\frac{1}{s}F(s(\phi_0 + v), \rho, \alpha), & s \ne 0, \\
D_uF(0, \rho, \alpha)(\phi_0 + v), &s = 0.
\end{cases}
\end{equation*}
Ekkor a megoldandó egyenletünk
\begin{equation}
G(v, \rho, \alpha, s) = 0
\end{equation}
alakban van, amit $v, \rho, \alpha$-ra kellene megoldani $s$ függvényében $0 \in \mathbb{R}$ egy környezetében. $G$ folytonosan differenciálható, és
\begin{equation*}
G(v, \rho, \alpha, 0) = D_uF(0, \rho, \alpha)(\phi_0 + v) = (\phi_0 + v)' + \rho D_uf(0, \alpha)(\phi_0 + v),
\end{equation*}
illetve
\begin{equation}
G(0, \rho, \alpha, 0) = \phi_0' + \rho D_uf(0, \alpha) \phi_0.
\end{equation}
Az implicit függvény tétel alkalmazásához arra van szükségünk, hogy a $(v, \rho, \alpha) \mapsto G(v, \rho, \alpha, s)$ leképezés $(0, 1, \alpha_0, 0)$-ban vett deriváltja folytonos és bijektív leképezése $V \times \mathbb{R} \times \mathbb{R}$-nak $Y$-ra. \par
Vegyük az előbbi egyenlet Taylor-sorba fejtését $(0, 1, \alpha_0)$ körül (rögzített $s \in \mathbb{R}$ esetén):
\begin{align*}
G(v, \rho, \alpha, s) = &G(0, 1, \alpha_0, s) + D_\rho G(0, 1, \alpha_0, s)(\rho - 1) \\
+ &D_\alpha G(0, 1, \alpha_0, s)(\alpha - \alpha_0) + D_vG(0, 1, \alpha_0, s)v + \dots,
\end{align*}
amit $s = 0$-ban kiértékelünk:
\begin{align*}
D_{v, \rho, \alpha}G(0, 1, \alpha_0, 0)(v, \rho - 1, \alpha - \alpha_0) = & D_uf(0, \alpha_0) \phi_0 (\rho - 1) \\
+ & D_{u \alpha}f(0, \alpha_0) \phi_0 (\alpha - \alpha_0) \\
+ & (v' + D_uf(0, \alpha_0)v).
\end{align*}
Ebből az utóbbi
\begin{equation*}
v \mapsto v' + D_uf(0, \alpha_0)v
\end{equation*}
lineáris homeomorfizmus $V$-ből $T$-re (folytonos bijekció), így elég az egyenlet maradékáról
\begin{equation*}
(\rho, \alpha) \mapsto D_uf(0, \alpha_0) \phi_0 (\rho - 1) + D_{u \alpha}f(0, \alpha_0) \phi_0 (\alpha - \alpha_0)
\end{equation*}
belátni hogy pontosan akkor képez $T$-be, ha $\rho = 1$ és $\alpha = \alpha_0$, valamint minden $\psi \in Z$-re egyértelműen létezik $(\rho, \alpha)$ amivel
\begin{equation*}
D_uf(0, \alpha_0) \phi_0 (\rho - 1) + D_{u \alpha}f(0, \alpha_0) \phi_0 (\alpha - \alpha_0) = \psi.
\end{equation*}
$T$-t korábban felírtuk mint $\{\psi_0, \psi_1\}$-re ortogonális függvények halmaza, így
\begin{equation*}
D_uf(0, \alpha_0) \phi_0 (\rho - 1) + D_{u \alpha}f(0, \alpha_0) \phi_0 (\alpha - \alpha_0) \in T
\end{equation*}
pontosan akkor teljesül, ha
\begin{equation*}
\langle D_uf(0, \alpha_0) \phi_0, \psi_i \rangle (\rho - 1) + \langle D_{u \alpha}f(0, \alpha_0) \phi_0, \psi_i \rangle (\alpha - \alpha_0) = 0 \qquad (i = 0, 1).
\end{equation*}
Mivel $D_uf(0, \alpha_0) \phi_0 = \phi_1$ és $\langle \phi_i, \psi_j \rangle = \delta_{ij}$, ezért az alábbi két egyenletet kapjuk (ahol $(\rho - 1)$ és $(\alpha - \alpha_0)$ az ismeretlenek)
\begin{align*}
\langle D_{u \alpha}f(0, \alpha_0) \phi_0, \psi_0 \rangle (\alpha - \alpha_0) &= 0 \\
(\rho - 1) + \langle D_{u \alpha}f(0, \alpha_0) \phi_0, \psi_1 \rangle (\alpha - \alpha_0) &= 0,
\end{align*}
aminek pontosan
\begin{equation*}
\langle D_{u \alpha}f(0, \alpha_0) \phi_0, \psi_0 \rangle \ne 0
\end{equation*}
esetén van csak a triviális megoldása (amikor is $\rho = 1$ és $\alpha = \alpha_0$), ami kiszámítható, hogy
\begin{equation*}
\langle D_{u \alpha}f(0, \alpha_0) \phi_0, \psi_0 \rangle = \Re (\frac{d\beta}{d\alpha}(\alpha_0)) \ne 0
\end{equation*}
a kezdeti kikötéseink miatt teljesül. \par
% példa
Vizsgáljunk meg példát, amint az előbbi levezetés alkalmazható. Vegyünk a
\begin{equation}
x'' + x - \alpha (1 - x^2) x' = 0
\end{equation}
differenciálegyenlet-rendszert. Alakítsuk át
\begin{equation*}
u =\left(
\begin{array}{c}
u_1 \\
u_2
\end{array}
\right) = \left(
\begin{array}{c}
x \\
x'
\end{array}
\right)
\end{equation*}
alakra, amiből az eredeti egyenlet
\begin{equation*}
u' + \left(
\begin{matrix}
&0& -1& \\
&1& -\alpha&
\end{matrix}
\right) + \left(
\begin{array}{c}
0 \\
u_1^2 u_2
\end{array}
\right) = \left(
\begin{array}{c}
0 \\
0
\end{array}
\right)
\end{equation*}
formában adódik. Így
\begin{equation*}
f(u, \alpha) = \left(
\begin{matrix}
&0& -1& \\
&1& -\alpha&
\end{matrix}
\right) + \left(
\begin{array}{c}
0 \\
u_1^2 u_2
\end{array}
\right)
\end{equation*}
illetve
\begin{equation*}
D_uf(0, \alpha) = \left(
\begin{matrix}
&0& -1& \\
&1& -\alpha&
\end{matrix}
\right),
\end{equation*}
aminek sajátértékei kielégítik az
\begin{equation*}
\beta (\alpha + \beta) + 1 = 0
\end{equation*}
egyenletet.
Legyen $\alpha_0 = 0$, így $\beta(0) = \pm i$ és $\beta' = -\frac{1}{2}$ az $\alpha = 0$ esetben, tehát a tételünk értelmében van $\eta > 0$ és $\alpha(s)$, $\rho(s)$ ($s \in (-\eta, \eta)$) amire $\alpha(0) = 0$, $\rho(0) = 1$ és $s \ne 0$ esetén van nemtriviális $x(s)$ megoldása a feladatunknak $2 \pi \rho(s)$ periódussal.

% TODO parcdiffegyenletek megoldása: nemlineáris peremérték feladat, szemilineáris elliptikus feladat

% TODOtalán Matlab program

%% 
 
%  Hivatkozások
\bibliography{hivatkozasok}	% TODO pontos adatok
\bibliographystyle{ieeetr}
 
\end{document}
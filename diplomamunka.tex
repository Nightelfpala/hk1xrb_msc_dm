\documentclass[oneside, titlepage, 12pt, a4paper]{report}

\usepackage{amsmath}	% align
\usepackage{amsfonts}	% \mathbb
\usepackage{amssymb}	% \mathcal
\usepackage{mathtools}	% \norm definiálás
\usepackage[magyar]{babel}	% tartalomjegyzék felirat
\usepackage[utf8]{inputenc}	% magyar karakterek
\usepackage[T1]{fontenc}
\usepackage{graphicx}	% ábrák
\usepackage{setspace}	% sorköz \onehalfspacing

\newtheorem{theorem}{Tétel}[section]
\newtheorem{lemma}{Lemma}[section]
\newtheorem{definition}{Definíció}[section]
\newtheorem{statement}{Állítás}[section]

\DeclareMathOperator{\Ima}{Im}	% képtér operátor
\DeclareMathOperator{\Ker}{Ker}	% magtér operátor
\DeclareMathOperator{\Dom}{Dom}	% értelmezési tartomány operátor
\DeclareMathOperator{\coker}{coker}	% magtér operátor
\DeclareMathOperator{\codim}{codim}	% kodimenzió operátor
\DeclareMathOperator{\ind}{ind}	% index operátor
\DeclareMathOperator{\Span}{span}	% kifeszített tér

\DeclarePairedDelimiter\norm{\lVert}{\rVert}	% norma

\textwidth=6.truein \textheight=9.truein
\hoffset=-.5truein
\voffset=-.8truein

\begin{document}
\begin{titlepage}
% belső fedőlap

% minta forrása: https://github.com/shdnx/ELTE-LaTeX-Thesis-Base
\begin{minipage}{0.40\linewidth}
\includegraphics[scale=0.8]{./abrak/elte_logo_szines.jpg}
\end{minipage}
\begin{minipage}{0.50\linewidth}
\begin{center}
Eötvös Loránd Tudományegyetem \\
Informatikai Kar \\
Numerikus Analízis Tanszék
\end{center}
\end{minipage}

\hrule
\vfill

\begin{center}
\Huge
\textbf{A Ljapunov-Schmidt-módszer}
\normalsize
\end{center}

\vfill

\begin{minipage}[t]{0.5\linewidth}
\begin{flushleft}
\textbf{Dr. Kovács Sándor} \\
Adjunktus
\end{flushleft}
\end{minipage}
\begin{minipage}[t]{0.5\linewidth}
\begin{flushright}
\textbf{Lipták Bence Gábor} \\
Programtervező Informatikus MSc
\end{flushright}
\end{minipage}

\vfill

\begin{center}
Budapest, 2018.
\end{center}

\end{titlepage}

\tableofcontents

%% Bevezetés

\onehalfspacing
\chapter{Bevezetés}
\label{chap:Introduction}

% TODO bevezetés szövege

%% Alapozás: faktorterek, Fredholm operátorok

\onehalfspacing
\chapter{Funkcionálanalízis kiegészítés}	% TODOtalán cím
\label{chap:Funcanal_ext}

Ahhoz, hogy a Ljapunov-Schmidt-módszert ismertethessük szükségünk van a Fredholm-operátorok fogalmára, amihez elengedhetetlenek a faktorterek és a kompakt operátorok. A módszerhez emellett még az implicit függvény tétel (Banach-terekben) is szükséges.

TODO jelölések szekció: Lin op, korl Lin op halmazai, esetleg skaláris szorzás jelölése, ha kell

TODO ebből mi maradjon meg? - kinek a szintjére kell belőni a részletességet, szükséges-e az, hogy pl egy évfolyamtárs megérthesse belőle az egészet?
faktortér def talán szükséges, Fredholm-op valószínűleg, Frechét-differenciálás és implicit fv tétel szintén

%
\section{Faktorterek}
\label{sec:Quotient_space}

Először is ismertessük a faktorterek definícióját, és az alkalmazásunk szempontjából fontos tulajdonságait, a \cite{faktorter} könyv 4.2 fejezete alapján, ahol a bizonyítások is megtalálhatóak.
\begin{definition}[Faktortér]
Legyen V egy $\mathbb{K}$ feletti lineáris tér, $U \subset V$ pedig egy altere. A V tér U szerinti \textbf{faktortere} vagy \textbf{hányadostere}
\begin{equation}
V / U := \{v + U \mid v \in V\},
\end{equation}
ahol
\begin{equation}
v + U := \{v + u \mid u \in U \}.
\end{equation}
\end{definition}

Így egy lineáris térben egy altér segítségével definiáltunk egy halmazrendszert. A következő állítással megfogalmazzuk, hogy a kapott halmazok és az U altér között mi az összefüggés.
\begin{statement}
Ha V egy $\mathbb{K}$ feletti lineáris tér, $U \subset V$ egy altere, akkor $\forall v, v' \in V$-re
\begin{equation}
v + U = v' + U \qquad \Leftrightarrow \qquad v - v' \in U.
\end{equation}
\end{statement}

Ennek segítségével belátható, hogy ha  ''$v - v' \in U$'' feltétellel definiálunk egy relációt $V$ elemein, akkor az egy ekvivalenciareláció és az ekvivalenciaosztályok pedig a $V / U$ faktortér elemei. Ezután definiáljunk műveleteket a faktortéren.
\begin{theorem}
Legyen V egy $\mathbb{K}$ feletti lineáris tér, $U \subset V$ egy altere, ekkor
\begin{itemize}
\item
Megadunk
\begin{align*}
V / U \times V / U & \longrightarrow V / U \\
\mathbb{K} \times V / U & \longrightarrow V / U
\end{align*}
leképezéseket a következő módon
\begin{align}
(v_1 + U) + (v_2 + U) &:= (v_1 + v_2) + U \\
\alpha (v + U) &:= \alpha v + U
\end{align}
melyek jól definiáltak.
\item
A $V / U$ faktortér ezekkel a műveletekkel egy $\mathbb{K}$ feletti lineáris tér.
\item
$\pi_U$ az úgynevezett \textbf{kanonikus leképezés}, melyet
\begin{equation*}
\pi_U(v) := v + U \qquad (v \in V)
\end{equation*}
módon definiálunk lineáris operátor $V$ és $V / U$ között.
\end{itemize}
\end{theorem}

A faktortér konstrukciója nagyon hasonlít az egész számok körében létesített maradékosztályokra, és rögzített $U \subset V$ esetén ugyanaz a jelölés is alkalmazható:
\begin{equation*}
\overline{v} := v + U \qquad (v \in V),
\end{equation*}
ezzel a jelöléssel a műveletek:
\begin{align*}
\overline{v} + \overline{w} &= \overline{v + w} &(v, w \in V) \\
\alpha \overline{v} &= \overline{\alpha v} &(\alpha \in \mathbb{K}, v \in V).
\end{align*}

Még fontos észrevétel, hogy a $\pi_U$ leképezés magtere pontosan az $U$ halmaz, valamint az operátor szürjektív is, amiből kapunk a faktortér dimenziójára egy összefüggést:
\begin{theorem}
Legyen V egy $\mathbb{K}$ feletti lineáris tér, $U \subset V$ egy altere, ekkor
\begin{equation}
\dim V / U + \dim U = \dim V.
\end{equation}
\end{theorem}

Ha egy $v \in V$ elemet egy $u \in U$ elemmel eltolunk ($U \subset V$ altér), akkor a $V / U$ faktortérbeli $\pi_U$ általi képe változatlan, ez a konstrukció alkalmas arra, hogy olyan függvényeket vizsgáljunk, amiknek az értéke az U altéren konstans, speciális esetben 0. Az ilyen leképezésekről szól a következő tétel:	% TODOtalán átfogalmazni
\begin{theorem}[Homomorfiatétel vektorterekre]
Legyen V egy $\mathbb{K}$ feletti lineáris tér, $F : V \rightarrow W$ egy lineáris leképezés, $U \subset V$ egy altér amire $U \subset \Ker F$. Ekkor egyértelműen létezik $F' : V / U \rightarrow W$ amivel $F = F' \circ \pi_U$. Emellett
\begin{itemize}
\item
$\Ima F = \Ima F'$, illetve F' pontosan akkor szürjektív, amikor F is,
\item
$\Ker F' = (\Ker F) / U$, illetve F' pontosan akkor injektív, amikor $U = \Ker F$.
\end{itemize}
F'-t az \textbf{F által indukált homomorfizmusnak} nevezzük.	% TODOmagyar név helyes-e
\end{theorem}

A faktortérnek van egy hasznos tulajdonsága, amivel nem halmazrendszerként, hanem altérként lehet kezelni.
\begin{theorem}
Legyen V egy $\mathbb{K}$ feletti lineáris tér, $U \subset V$ egy altere, $W \subset V$ pedig az U komplemens altere (tehát $V = W \oplus U$). Ekkor a $\pi_U$ kanonikus leképezés leszűkítése W-re
\begin{equation*}
\pi_{\mkern 1mu \vrule height 2ex\mkern2mu U} : W \longrightarrow V / U, \; \pi_{\mkern 1mu \vrule height 2ex\mkern2mu U}(w) = w + U
\end{equation*}
izomorfia, azaz $W \cong V / U$.
\end{theorem}

% TODOtalán kivezető szöveg

% TODOtalán 4.2.8 tétel ~ ennek analógja funkcionálokra (ha ezt később használjuk)

%
\section{Kompakt operátorok}
\label{sec:kompakt}

A Fredholm-operátorok konstrukciójához érdemes feleleveníteni a kompakt operátorok fogalmát és néhány, a gyakorlat szempontjából hasznos tulajdonságukat. A következők során $(X, \norm{.}_X)$ és $(Y, \norm{.}_Y)$ normált terek.
\begin{definition}[Kompakt operátor]
$A : X \rightarrow Y$ operátor \textbf{kompakt}, ha bármely $U \subset X$ korlátos halmaznak a képe prekompakt, azaz $\overline{A[U]} \subset Y$ kompakt, valamint ha A korlátos, akkor \textbf{teljesen folytonosnak} is nevezzük. \cite{funkanal}
\end{definition}

A továbbiakban az $X \rightarrow Y$ közötti kompakt és lineáris operátorok halmazát $K(X, Y)$-al, $X = Y$ esetén $K(X)$-szel jelöljük, ez zárt alteret alkot $L(X, Y)$-ban.
\begin{definition}
$A \in L(X, Y)$ operátor \textbf{véges rangú}, ha a képtere véges dimenziós ($\dim \Ima A < \infty$). A véges rangú operátorok halmazát $K_{fin}(X, Y)$-al rövidítjük.
\cite{funkanal,FcNotex}
\end{definition}
Belátható (\cite{funkanal}), hogy minden véges rangú operátor kompakt (és így a jelölés indokolt). \par

A kompakt operátorok bizonyos esetekben közelíthetőek véges rangú operátorokkal:
\begin{theorem}
Ha Y-ban van Schauder-bázis, akkor $A \in L(X, Y)$ pontosan akkor kompakt, ha határértéke véges rangú operátorok valamely sorozatának, azaz $K(X, Y) = \overline{K_{fin}(X, Y)}$. \cite{FcNotex}
\end{theorem}
Ezek a feltételek teljesülnek például szeparábilis Hilbert-terekben vagy $L^p$-terekben ($p \ge 1$). \par

Még tegyünk egy megállapítást a kompakt operátorok adjungáltjával kapcsolatban:
\begin{theorem}
Legyenek $(X, \norm{.}_X)$ és $(Y, \norm{.}_Y)$ Banach-terek, $A \in L(X, Y)$, ekkor
\begin{equation*}
A \in K(X, Y) \Leftrightarrow A^* \in K(Y^*, X^*),
\end{equation*}
azaz A pontosan akkor kompakt, ha $A^*$ adjungált operátora kompakt.
\cite{funkanal, lectures16and17}
\end{theorem}
% TODOtalán kivezető szöveg

%
\section{Fredholm-operátorok}
\label{sec:Fredholm}

Végül beszéljünk a Fredholm-operárokról és lehetőségekről az előállításukra. A továbbiakban $(X, \norm{.}_X)$ és $(Y, \norm{.}_Y)$ Banach-terek.
\begin{definition}[Fredholm-operátor]
$T \in L(X, Y)$ \textbf{Fredholm-operátor}, ha az alábbiak teljesülnek:
\begin{itemize}
\item
$\dim \Ker T < \infty$,
\item
T[X] zárt Y-ban,
\item
$\dim (Y / T[X]) < \infty$.
\end{itemize} % TODOmagyar nevek
Ekkor a T operátor \textbf{indexe} $\ind T := \dim \Ker T - \dim (Y / T[X]) \in \mathbb{Z}$, T \textbf{cokernele} $\coker T := Y / T[X]$ (azaz Y-ban a T képtere szerinti faktortér), valamint a cokernel dimenziója az operátor \textbf{kodimenziója} $\codim T := \dim \coker T$. \cite{faktorter, FcNotex, lectures16and17, diffun2}
\end{definition}
Az X és Y közötti Fredholm-operátorok halmazát a továbbiakban $\mathcal{F}(X, Y)$-al jelöljük. Belátható, hogy a második feltétel (T képterének a zártsága) a másik kettőből következik \cite{lectures16and17, diffun2}. \par

Most nézzük meg, hogy bizonyos függvény műveletek hatására változik-e a Fredholm-tulajdonság.	% TODOtalán átírni
\begin{theorem}
Legyenek $(X, \norm{.}_X)$, $(Y, \norm{.}_Y)$ és $(Z, \norm{.}_Z)$ Banach-terek, ekkor:
\begin{itemize}
\item
$A \in \mathcal{F}(X, Y)$ és $B \in \mathcal{F}(Y, Z)$, ekkor $B \circ A \in \mathcal{F}(X, Z)$ és $ \ind (B \circ A) = \ind B + \ind A$,
\item
$A \in \mathcal{F}(X, Y)$, ekkor $A^* \in \mathcal{F}(X^*, Y^*)$ és $\ind A^* = - \ind A$,
\item
$\mathcal{F}(X, Y)$ nyílt részhalmaza L(X, Y)-nak, és az $\ind : \mathcal{F}(X, Y) \rightarrow \mathbb{Z}$ függvény lokálisan konstans.
\end{itemize}
Tehát Fredholm-operátorok kompozíciója Fredholm-operátor, valamint Fredholm-operátor adjungáltja is Fredholm-operátor. \cite{FcNotex, diffun2}
\end{theorem}
% TODOtalán a lokálisan konstansságból következően a kis perturbáció tétel is?

Korábban említettük a kompakt operátorok és a Fredholm-operátorok kapcsolatát, erről szól a következő állítás.
\begin{theorem}
$T \in L(X, Y)$ bijektív, $K \in L(X, Y)$ kompakt, ekkor $T + K$ Fredholm-operátor és $\ind(T + K) = 0$. Ennek speciális esete amikor $(X, \norm{.}_X) = (Y, \norm{.}_Y)$ és T az identitás X-en. \cite{FcNotex,diffun2}
\end{theorem}

Amennyiben ilyen módon kaptunk egy Fredholm-operátort, akkor ahhoz (a kompakt operátorok altér-tulajdonságából kifolyólag) egy kompakt operátort hozzáadva is Fredholm-operátort kapunk, ez igaz tetszőleges konstrukció esetén is:
\begin{theorem}
$K \in K(X, Y)$ és $F \in \mathcal{F}(X, Y)$ esetén $K + F \in \mathcal{F}(X, Y)$, valamint $\ind(K + F) = \ind F$. \cite{FcNotex}
\end{theorem}

Mivel a Fredholm-operátoroknál nem feltétel, hogy a magterük csak a tér nullelemét tartalmazza, ezért általában nem invertálhatóak, de egy hasonló tulajdonságot megadhatunk:	% TODOtalán átírni
\begin{theorem}
$T \in L(X, Y)$ pontosan akkor Fredholm-operátor, ha létezik $B \in L(Y, X)$, $K_X \in K(X)$, $K_Y \in K(Y)$ úgy, hogy
\begin{equation*}
BT = I_{\mkern 1mu \vrule height 2ex\mkern2mu X} + K_X, TB = I_{\mkern 1mu \vrule height 2ex\mkern2mu Y} + K_Y.
\end{equation*}
Azaz a Fredholm-tulajdonság lényegében az invertálhatóság modulo kompakt operátort jelenti. \cite{FcNotex, diffun2}
\end{theorem}
% TODOtalán kivezető szöveg

%
\section{Implicit függvény tétel}
\label{sec:implicitfvtetel}

A módszer használata során az eredmény eléréséhez szükségünk lesz az implicit függvény tételre Banach-terekben, illetve ahhoz kötődően a Fréchet-derivált fogalmára. $(X, \norm{.}_X)$, $(Y, \norm{.}_Y)$ és $(Z, \norm{.}_Z)$ a következők során Banach-terek.

\begin{definition}
$F : X \times Y \rightarrow Z$ \textbf{Fréchet-differenciálható} X-ben az $(u_0, v_0)$ pontban, ha létezik $(D_xF)(u_0, v_0) \in L(X, Z)$ úgy, hogy
\begin{equation*}
\lim_{h \to 0} \frac{\norm{F(u_0 + h, v_0) - F(u_0, v_0) - (D_xF)(u_0, v_0)h}_Z}{\norm{h}_X} = 0.
\end{equation*}
\end{definition}
Látható, hogy egy ponthoz tartozó derivált (a valós, többdimenziós esethez hasonlóan) egy függvény.

\begin{theorem}[Implicit függvény tétel Banach-terekben]
\label{implicit}
$F : X \times Y \rightarrow Z$ folytonos, $(u_0, v_0) \in X \times Y$, $F(u_0, v_0) = 0$, $(D_xF)(u_0, v_0)$ bijektív és folytonos. Ekkor létezik $(u_0, v_0)$-nak olyan $U \times V \subset X \times Y$ környezete és $G:V \rightarrow U$ függvény, amivel $G(v_0) = u_0$ és
\begin{equation*}
F(G(v), v) = 0 \qquad (\forall v \in V).
\end{equation*}
Ezen felül minden $U \times V$-beli megoldás ebben a formában áll elő. \cite{IFaLS}
\end{theorem}
Az ilyen tulajdonságokkal rendelkező $D_xF$ függvényeket lineáris homeomorfizmusnak nevezzük:
\begin{definition}
Az $A$ leképezés \textbf{lineáris homeomorfizmus}, ha folytonos, bijektív és az inverze is folytonos.
\end{definition}
Az inverz folytonossága pedig a Banach-féle inverz tételből (vagy Banach-féle homeomorfia-tételből) következik:
\begin{theorem}
$A \in L(X, Y)$ Banach-terek közötti bijektív operátor, ekkor $A^{-1} \in L(Y, X)$. (\cite{funkanal} 6.1.4) % TODO kell-e pontos helymegjelölés
\end{theorem}
% TODOtalán kivezető szöveg
% TODOtalán Taylor sorfejtés Banach-térben

%%

\onehalfspacing
\chapter{A Ljapunov-Schmidt-módszer és alkalmazásai}
\label{chap:LjapunovSchmidt}

Először is vizsgáljunk meg egy bifurkációs problémát \cite{CSiAM} alapján, ezen keresztül szemléltetve a módszer lényegét. Tekintsük a következő egyenletet:
\begin{equation*}
F(\lambda, x) = 0
\end{equation*}
ahol $\lambda \in \mathbb{R}$ valós paraméter, $x \in X$ állapotváltozó (TODO ?) $X$ Banach-térben, $0 \in Y$ pedig egy Banach-tér nulleleme, $F$ pedig kétszer folytonosan differenciálható operátor. A feladat meghatározni azon $(\lambda, x) \in \mathbb{R} \times X$ párokat, amelyek kielégítik az egyenletet, lehetőség szerint az $x$-eket $\lambda$ függvényében.

Feltesszük, hogy létezik megoldás, valamint azt, hogy minden $x = 0$ esetén minden $\lambda \in \mathbb{R}$ megoldása az egyenletnek (az úgynevezett triviális megoldások). Ezen kívül tegyük fel azt, hogy $(\lambda_0, 0) \in \mathbb{R} \times X$ bármely környezetében van nemtriviális megoldás, azaz $(\lambda_0, 0)$ bifurkációs pont. Ez maga után vonja, hogy $F_x(\lambda_0, 0)$ Fréchet-derivált nem invertálható.

Legyen
\begin{align*}
L &:= F_x(\lambda_0, x_0) : X \rightarrow Y, \\
K &:= \Ker L, \\
R &:= \Ima L.
\end{align*}
Tegyük fel, hogy $K$-nak és $R$-nek (amelyek zárt alterek $X$-ben és $Y$-ban) vannak komplemens alterei, azaz létezik $W \subset X$ zárt altér, amellyel $K \oplus W = X$, illetve $Z \subset Y$ szintén zárt altér, amellyel $R \oplus Z = Y$, és bármely $x \in X$ egyértelműen felírható $x = u + v, u \in K, v \in W$ alakban, valamint bármely $y \in Y$ egyértelműen felírható $y = r + z, r \in R, z \in Z$ alakban - ezek teljesülnek például akkor, hogyha $K$ és $R$ véges dimenziós alterek, azaz ha $L$ Fredholm-operátor. Vegyük ezenfelül a $Q : Y \rightarrow R$ és $P : Y \rightarrow Z$ projekciókat.

Írjuk fel az eredeti egyenlet Taylor-polinomját:
\begin{equation*}
0 = F(\lambda, x) = Lx + \phi (\lambda, x)
\end{equation*}
(ahol $\phi(\lambda, x) = F(\lambda, x) - Lx$ a megfelelő maradéktag), és ezekbe írjuk be az $x = u + v$ felírást, valamint vetítsük őket az $R$ és a $Z$ alterekre, így kapjuk az alábbi két egyenletet:
\begin{align*}
0 &= QL(u + v) + Q\phi (\lambda, u + v) = Lv + Q\phi (\lambda, u + v), \\
0 &= PL(u + v) + P\phi (\lambda, u + v).
\end{align*}
Az első egyenlet így egy 3-változós függvényt ír le:
\begin{equation*}
\Phi(\lambda, u, v) := Lv + Q\phi (\lambda, u + v),
\end{equation*}
ami folytonosan differenciálható, és deriváltja a 3. változó szerint az $u = v = 0$ helyen
\begin{equation*}
\Phi_v (\lambda_0, 0, 0) : v \rightarrow Lv + Q\phi_x(\lambda_0, 0)v.
\end{equation*}
Mivel $\phi(\lambda, x) = F(\lambda, x) - Lx$, így
\begin{equation*}
\phi_x(\lambda_0, 0) = F_x(\lambda_0, 0) - L = L - L = 0,
\end{equation*}
ezért
\begin{equation*}
\Phi_v (\lambda_0, 0, 0) = L_{\mkern 1mu \vrule height 2ex\mkern2mu W},
\end{equation*}
viszont $\Ker L_{\mkern 1mu \vrule height 2ex\mkern2mu W} = \{0\}$ és $\Ima L = R$, így $\Phi_v (\lambda, 0, 0)$ folytonos bijekció $W$ és $R$ között. Alkalmazható az implicit függvény tétel, tehát van $(\lambda_0, 0, 0)$-nak egy olyan $\Lambda \times \mathcal{K} \times \mathcal{W}$ környezete, amiben egy $\gamma : \Lambda \times \mathcal{K} \rightarrow \mathcal{W}$ függvény meghatározza a $\Phi_v (\lambda, u, v) = 0$ összes megoldását $\Phi_v (\lambda, u, \gamma(\lambda, u))$ alakban.
Ezt behelyettesítve az eredeti egyenletbe kapjuk az
\begin{equation*}
0 = P F(\lambda, u + \gamma(\lambda, u))
\end{equation*}
egyenletet. Mivel $u \in K$ és $\dim K < \infty$, valamint $\Ima P = Z$, $\dim Z < \infty$, így az eredeti egyenletet sikerült redukálnunk egy véges dimenzión értelmezett, véges dimenziós értékkészletű (TODO?) egyenletre, amit könnyebb megoldani.


\onehalfspacing
\chapter{A Ljapunov-Schmidt-módszer mint numerikus módszer}
\label{chap:numeric}

A következő fejezetben ismertetjük a Ljapunov-Schmidt módszernek egy numerikus módszerkénti felhasználását bizonyos peremérték-feladatok esetén \cite{LSNum} alapján. % TODO irodalomjegyzékben rossz kis-nagybetűk a címnél
Tekintsük az alábbi egyenletet:
\begin{equation}
Lu(x) = Nu(x),\;x\in[a, b] \label{eq:num:1}
\end{equation}
ahol $L$ úgynevezett Sturm-Liouville operátor:
\begin{equation*}
Lu(x) = \frac{1}{w(x)}(-(p(x)u'(x))' + q(x)u(x)),
\end{equation*}
ahol $p(x) > 0, w(x) > 0\;(x \in [a, b])$, $p \in C^1[a, b];\;q, w \in C[a, b]$ adott függvények, valamint $u$-ra a következő peremfeltételek teljesülnek:
\begin{align*}
\alpha_{11}u(a) + \alpha_{12}u'(a) &= 0, \\
\alpha_{21}u(b) + \alpha_{22}u'(b) &= 0.
\end{align*}
Ezek a kikötések azért szükségesek, mert a későbbiekben az ilyen $L$-ekhez adunk meg egy módszert (a Csebisev-Tau-módszert \cite{ChebysevTau}), amivel a sajátfüggvényeit elő tudjuk állítani. \\ % TODO irodalomjegyzék kis-nagybetű
Emellett feltesszük a következőket:
\begin{itemize}
\item $S$ valós, szeparábilis Hilbert-tér, $L : \Dom L \subset S \rightarrow S$ lineáris operátor, $N : \Dom N \subset S \rightarrow S$ nemlineáris operátor,
\item $L$ zárt leképezés (amivel ekvivalens \cite{funkanal}: $x_n \rightarrow x$ és $Lx_n \rightarrow y$-ból következik hogy $x \in \Dom L$ és $Lx = y$), önadjungált, $\Dom L$ sűrű $S$-ben, $\Ker L = p > 0$ véges,
\item $L$-nek a sajátértékei $\lambda_1 = \dots = \lambda_p = 0 < \lambda_{p + 1}$, $\lambda_i \leq \lambda_{i + 1}$, $\lim_{i \to \infty} \lambda_i = \infty$, és a hozzájuk tartozó $\Phi_1, \Phi_2, \dots$ sajátfüggvények $S$-ben egy teljes ortonormált rendszert alkotnak - ezeket a Sturm-Liouville-operátor biztosítja, % TODO lim legyen alatta, kell-e a S-L tulajdonság említése
\item létezik egy $S' \subset S$ altér, ami egy $\mu$ normával teljes, $\Dom L \subset S'$, minden $x \in \Dom L$ esetén a $\Phi_i$-szerinti Fourier-sora ($\sum_{k=1}^\infty (x, \Phi_k) \Phi_k$) $\mu$-ben konvergál $x$-hez és $\{ \mu(\Phi_k) / \lambda_k\}_{k > p} \in l_2$, valamint létezik $\alpha > 0$, amivel $x \in S'$ esetén $\norm {x}_S \leq \alpha \mu(x)$ (tehát $\mu$ $\alpha$-szorosa felső becslése az eredeti normának, és $\mu$-beli konvergenciából következik, hogy az adott sorozat az eredeti normában is konvergens),
\item $\Dom L \cap \Dom N \neq \emptyset$, $\Dom N \subset S'$ és altér $S'$-ben, $\Dom N$ zárt $\mu$ szerint,
\item bármely $R > 0$-hoz létezik $\beta_R > 0, b_R > 0$, amelyekkel bármely $x, y \in \Dom N$ amire $\mu(x) \leq R, \mu(y) \leq R$ teljesül $\mu(Nx - Ny) \leq \beta_R \mu(x - y)$ és $\mu(Nx) \leq b_R$, tehát tetszőleges $\mu$-szerinti gömb $N$-szerinti képe korlátos ($\mu$ normával), valamint minden elem környezetének képének átmérője arányos a környezet átmérőjével; ezek a feltételek biztosítják majd az kisegítő egyenletünk kontrakció tulajdonságát. % TODO átfogalmazni a környezeteset % TODO auxiliary = kisegítő?
\end{itemize}
Az $\eqref{eq:num:1}$ egyenlet megoldásait $\Dom L \cap \Dom N$-ben keressük. Legyen $m \geq p$ és
\begin{equation*}
S_m = \Span \{\Phi_1, \dots, \Phi_m \}, \; S_0 = \{ 0\}
\end{equation*}
az első $m$ sajátfüggvény által kifeszített altér, $S_m \subset \Dom L$. Definiáljuk a következő operátorokat, amennyiben $u = \sum_{k = 1}^\infty (u, \Phi_k), \Phi_k$ ($u \in S$):
\begin{equation*}
P_m u := \sum_{k = 1}^m (u, \Phi_k), \Phi_k,
\end{equation*}
tehát $P_m$ az $S_m$-re történő ortogonális projekció, valamint
\begin{equation*}
H_m u := \sum_{k = m + 1}^\infty \frac{1}{\lambda_k} (u, \Phi_k), \Phi_k.
\end{equation*}
$H_m$-ről látható, hogy lineáris, $\Dom L$-be képez, $H_m = (L_{\mkern 1mu \vrule height 2ex\mkern2mu S_m^\perp})^{-1}$, illetve $H_mLu = (I - P_m)u$ ($I$ az identitás operátor). Az $S$-téren vett normája $\norm{H_m} = \frac{1}{\lambda_{m+1}}$, a $\mu$-szerinti normája $\mu(H_m) \leq \alpha \sigma(m)$, ahol $\sigma(m) = \left(\sum_{k = m+1}^\infty \left( \frac{\mu(\Phi_k)}{\lambda_k}\right)^2\right)^{1/2}$ (ez utóbbi Cauchy-Schwarz-egyenlőtlenséggel belátható). Ebből következik, hogy $\lim_{m \to \infty} \mu(H_m) = 0$. A korábbi feltételeinket egészítsük ki még azzal, hogy $\Ima H_m \subset \Dom N$ és $S_m \subset \Dom N$, hogy a $P_m$ és a $H_m$ alkalmazása után tudjuk az $N$-et is még alkalmazni. \\
Tegyük fel, hogy $\overline{u} \in \Dom N \cap \Dom L$ megoldása az $\eqref{eq:num:1}$ egyenletnek, tehát
\begin{equation}
L\overline{u} = N\overline{u}. \label{eq:num:2}
\end{equation}
Erre először alkalmazzuk $H_m$-et:
\begin{align}
H_m L \overline{u} &= H_m N \overline{u}, \nonumber \\
(I - P_m) \overline{u} &= H_m N \overline{u}, \nonumber \\
\overline{u} &= P_m \overline{u} + H_m N \overline{u}, \label{eq:num:3}
\end{align}
ez a kisegítő egyenlet. Ezután alkalmazzuk $\eqref{eq:num:2}$-re $P_m$-et: % TODO auxiliary = kisegítő?
\begin{equation}
P_m(L \overline{u} - N \overline{u}) = 0, \label{eq:num:4}
\end{equation}
ami a bifurkációs egyenlet. Az összes olyan $\overline{u} \in \Dom L \cap \Dom N$, ami megoldása a kisegítő és a bifurkációs egyenletnek, az megoldása az eredeti $\eqref{eq:num:1}$ feladatnak. \\ % TODO auxiliary = kisegítő?
Legyenek $a > 0, b > 0$ valós számok, és legyen $u_0$ egy közelítő megoldása az $Lu = Nu$ egyenletnek úgy, hogy létezik $u^* \in S_m$ ($u^* = \sum_{k=1}^m c_k \Phi_k$), amivel $\mu(u^* - u_0) \leq a$. Vegyük a következő halmazt:
\begin{equation*}
S_{u^*}^b := \{ u \in \Dom N \; | \; P_m u = u^*, \mu((I - P_m)u) \leq b \},
\end{equation*}
tehát $S_{u^*}^b$ azon $u$-kat tartalmazza, melyek $S_m$-re levetítve $u^*$-ba esnek, és $u^*$-tól $\mu$-szerinti távolságuk legfeljebb $b$. Ezen elemek $\mu$-normája korlátos:
\begin{equation*}
\mu(u) = \mu(P_m u + (I - P_m) u) \leq \mu(u^*) + \mu((I - P_m)u) \leq \mu(u^*) + b.
\end{equation*}
 Ezután definiáljuk a következő operátort:
\begin{align*}
T_{u^*}^b &: S_{u^*}^b \rightarrow S, \\
T_{u^*}^b (u) &:= u^* + H_m N u.
\end{align*}
Lássuk be $T_{u^*}^b$-ről, hogy bizonyos feltételek mellett kontrakció, legyen $x, y \in S_{u^*}^b$:
\begin{align*}
\mu( T_{u^*}^b(x) - T_{u^*}^b(y)) &= \mu((u^* + H_m N x) - (u^* + H_m N y)) = \\
&= \mu(H_m(N x - N y)) \leq \\
&\leq \mu(H_m) \mu(N x - N y) \leq \\
&\leq \mu(H_m) \beta_R \mu(x - y),
\end{align*}
ahol $R = \mu(u^*) + b$, tehát $u^*$-tól és $b$-től függő konstans, és a kezdeti kikötéseink alapján $\beta_R$ egy $R$-től függő konstans. Mivel $\lim_{m \to \infty} \mu(H_m) = 0$, ezért elég nagy $m$ esetén $\mu(H_m) \beta_R < 1$. Annak, hogy $\Ima  T_{u^*}^b \subset S_{u^*}^b$ (tehát lehet $ T_{u^*}^b$-t iteratívan alkalmazni) elégséges feltétele, hogy $\mu(H_m)^2 \mu(L) b_R \leq b$, ami kellően nagy $m$-re szintén teljesül. Tehát ha $m$ elég nagy, akkor $ T_{u^*}^b$ kontrakció, így a Banach-Tyihonov-Cacciopoli-tétel miatt van fixpontja. Ezt az $u^*$-tól függő fixpontot jelöljük $y(u^*)$-al, és asszociált elemnek nevezzük. $y(u^*)$-ról könnyen belátható, hogy megoldása az $\eqref{eq:num:3}$ egyenletnek. \\ % TODO associate = asszociált?
Vezessük be a $c_k := (u^*, \Phi_k) \;(k = 1, \dots, m)$ jelölést, amivel $u^* = \sum_{k = 1}^m c_k \Phi_k$. Nézzük meg, hogy milyen feltételek mellett lesz $y(u^*)$ megoldása a bifurkációs egyenletnek $\eqref{eq:num:4}$?
\begin{align*}
0 &= P_m(L y(u^*) - N y(u^*)) = P_m L y(u^*) - P_m N y(u^*), \\
P_m L y(u^*) &= P_m L (u^* + H_m N y(u^*)) = P_m L u^* + P_m L H_m N y(u^*) = \\
 &= P_m L (\sum_{k = 1}^m c_k\Phi_k) + P_m L \sum_{k = m+1}^\infty (N y(u^*), \Phi_k) \Phi_k = \\
 &= P_m (\sum_{k = 1}^m c_k \lambda_k \Phi_k) + P_m \sum_{k = m + 1}^\infty (N y(u^*, \Phi_k) \lambda_k \Phi_k = \\
 &= P_m (\sum_{k = 1}^m c_k \lambda_k \Phi_k) + \sum_{k = m + 1}^\infty (N y(u^*), \Phi_k) \lambda_k P_m \Phi_k = \\
 &= P_m (\sum_{k = 1}^m c_k \lambda_k \Phi_k),
\end{align*}
tehát
\begin{equation*}
0 = P_m (\sum_{k = 1}^m c_k \lambda_k \Phi_k - N y(u^*)),
\end{equation*}
ami pontosan akkor teljesül, ha
\begin{align}
0 &= (\sum_{j = 1}^m c_j \lambda_j \Phi_j - N y(u^*), \Phi_k) \; (k = 1, \dots, m), \nonumber \\
0 &= (c_k \lambda_k \Phi_k - N y(u^*), \Phi_k) \; (k = 1, \dots, m), \text{vagy} \nonumber \\
0 &= (\lambda_k u^* - N y(u^*), \Phi_k) \; (k = 1, \dots, m). \label{eq:num:5}
\end{align}
Ez utóbbi egy $m$-változós ($c_k \; (k = 1, \dots, m)$ számok meghatározzák $u^*$-ot), $m$ egyenletből álló egyenletrendszer. Ezek eredményeként megállapíthatjuk, hogy ha $a, b, m$ elég nagyok, akkor az eredeti $\eqref{eq:num:1}$ egyenletnek $\overline{u}$ pontosan akkor megoldása, ha az $\eqref{eq:num:5}$ egyenletnek $u^*$ megoldása és $\overline{u} = y(u^*)$. \\



% TODO példa

% TODO Csebisev-Tau


% TODO kiegészítés: H_m tul levezetések


%% 
 
%  Hivatkozások
\bibliography{hivatkozasok}	% TODO pontos adatok, kis-nagybetű javítások a cikkekhez
\bibliographystyle{ieeetr}
 
\end{document}
\documentclass[oneside, titlepage, 12pt, a4paper]{report}

\usepackage{amsmath}	% align
\usepackage{amsfonts}	% \mathbb
\usepackage{mathtools}	% \norm definiálás
\usepackage[magyar]{babel}	% tartalomjegyzék felirat
\usepackage[utf8]{inputenc}	% magyar karakterek
\usepackage[T1]{fontenc}
\usepackage{graphicx}	% ábrák
\usepackage{setspace}	% sorköz \onehalfspacing

\newtheorem{theorem}{Tétel}[section]
\newtheorem{lemma}{Lemma}[section]
\newtheorem{definition}{Definíció}[section]
\newtheorem{statement}{Állítás}[section]

\DeclareMathOperator{\Ima}{Im}	% képtér operátor
\DeclareMathOperator{\Ker}{Ker}	% magtér operátor
\DeclareMathOperator{\Codim}{codim}	% kodimenzió operátor

\DeclarePairedDelimiter\norm{\lVert}{\rVert}	% norma

\textwidth=6.truein \textheight=9.truein
\hoffset=-.5truein
\voffset=-.8truein

\begin{document}
\begin{titlepage}
% belső fedőlap

% minta forrása: https://github.com/shdnx/ELTE-LaTeX-Thesis-Base
\begin{minipage}{0.40\linewidth}
\includegraphics[scale=0.8]{./abrak/elte_logo_szines.jpg}
\end{minipage}
\begin{minipage}{0.50\linewidth}
\begin{center}
Eötvös Loránd Tudományegyetem \\
Informatikai Kar \\
Numerikus Analízis Tanszék
\end{center}
\end{minipage}

\hrule
\vfill

\begin{center}
\Huge
\textbf{A Ljapunov-Schmidt-módszer}
\normalsize
\end{center}

\vfill

\begin{minipage}[t]{0.5\linewidth}
\begin{flushleft}
\textbf{Dr. Kovács Sándor} \\
Adjunktus
\end{flushleft}
\end{minipage}
\begin{minipage}[t]{0.5\linewidth}
\begin{flushright}
\textbf{Lipták Bence Gábor} \\
Programtervező Informatikus MSc
\end{flushright}
\end{minipage}

\vfill

\begin{center}
Budapest, 2018.
\end{center}

\end{titlepage}

\tableofcontents

%% Bevezetés

\onehalfspacing
\chapter{Bevezetés}
\label{chap:Introduction}

% TODO bevezetés szövege

%% Alapozás: faktorterek, Fredholm operátorok

\onehalfspacing
\chapter{Funkcionálanalízis kiegészítés}
\label{chap:Funcanal_ext}

Ahhoz, hogy a Ljapunov-Schmidt-módszert ismertethessük, szükségünk van a faktorterek és a Fredholm-operátorok fogalmaira.

% TODO esetleg jelölések szekció: Lin op, korl Lin op halmazai

\section{Faktorterek}
\label{sec:Faktorter}

Először is ismertessük a faktorterek definícióját, és az alkalmazásunk szempontjából fontos tulajdonságait, a \cite{faktorter} könyv 4.2 fejezete alapján, ahol a bizonyítások is megtalálhatóak.
\begin{definition}[Faktortér]
Legyen V egy $\mathbb{K}$ feletti lineáris tér, $U \subset V$ pedig egy altere. A V tér U szerinti \textbf{faktortere} vagy \textbf{hányadostere}
\begin{equation}
V / U := \{v + U \mid v \in V\},
\end{equation}
ahol
\begin{equation}
v + U := \{v + u \mid u \in U \}.
\end{equation}
\end{definition}

Így egy lineáris térben egy altér segítségével definiáltunk egy halmazrendszert. A következő állítással megfogalmazzuk, hogy a kapott halmazok és az U altér között mi az összefüggés.
\begin{statement}
Ha V egy $\mathbb{K}$ feletti lineáris tér, $U \subset V$ egy altere, akkor $\forall v, v' \in V$-re
\begin{equation}
v + U = v' + U \Leftrightarrow v - v' \in U.
\end{equation}
\end{statement}

Ennek segítségével belátható, hogy ha  ''$v - v' \in U$'' feltétellel definiálunk egy relációt $V$ elemein, akkor az egy ekvivalenciareláció és az ekvivalenciaosztályok pedig a $V / U$ faktortér elemei. Ezután definiáljunk műveleteket a faktortéren.
\begin{theorem}
Legyen V egy $\mathbb{K}$ feletti lineáris tér, $U \subset V$ egy altere, ekkor
\begin{itemize}
\item
Megadunk	% TODO átfogalmazás
\begin{align*}
V / U \times V / U & \longrightarrow V / U \\
\mathbb{K} \times V / U & \longrightarrow V / U
\end{align*}
leképezéseket a következő módon
\begin{align}
(v_1 + U) + (v_2 + U) &:= (v_1 + v_2) + U \\
\alpha (v + U) &:= \alpha v + U
\end{align}
melyek jól definiáltak.
\item
A $V / U$ faktortér ezekkel a műveletekkel egy $\mathbb{K}$ feletti lineáris tér.
\item
$\pi_U$ az úgynevezett \textbf{kanonikus leképezés}, melyet	% TODO magyar név helyes-e
\begin{equation*}
\pi_U(v) := v + U \; (v \in V)
\end{equation*}
módon definiálunk lineáris operátor $V$ és $V / U$ között.
\end{itemize}
\end{theorem}

A faktortér konstrukciója nagyon hasonlít az egész számok körében létesített maradékosztályokra, és rögzített $U \subset V$ esetén ugyanaz a jelölés is alkalmazható:
\begin{equation*}
\overline{v} := v + U \; (v \in V),
\end{equation*}
ezzel a jelöléssel a műveletek:
\begin{align*}
\overline{v} + \overline{w} &= \overline{v + w} \; &(v, w \in V) \\
\alpha \overline{v} &= \overline{\alpha v} \; &(\alpha \in \mathbb{K}, v \in V).
\end{align*}

Még fontos észrevétel, hogy a $\pi_U$ leképezés magtere pontosan az $U$ halmaz, valamint az operátor szürjektív is, amiből kapunk a faktortér dimenziójára egy összefüggést:
\begin{theorem}
Legyen V egy $\mathbb{K}$ feletti lineáris tér, $U \subset V$ egy altere, ekkor
\begin{equation}
\dim V / U + \dim U = \dim V.
\end{equation}
\end{theorem}

Ha egy $v \in V$ elemet egy $u \in U$ elemmel eltolunk ($U \subset V$ altér), akkor a $V / U$ faktortérbeli $\pi_U$ általi képe változatlan, ez a konstrukció alkalmas arra, hogy olyan függvényeket vizsgáljunk, amiknek az értéke az U altéren konstans, speciális esetben 0.	% TODO esetleg átfogalmazni
\begin{theorem}[Homomorfiatétel vektorterekre]
Legyen V egy $\mathbb{K}$ feletti lineáris tér, $F \, : \, V \rightarrow W$ egy lineáris leképezés, $U \subset V$ egy altér amire $U \subset \Ker F$. Ekkor egyértelműen létezik $F' \, : \, V / U \rightarrow W$ amivel $F = F' \circ \pi_U$. Emellett
\begin{itemize}
\item
$\Ima F = \Ima F'$, illetve F' pontosan akkor szürjektív, amikor F is,
\item
$\Ker F' = (\Ker F) / U$, illetve F' pontosan akkor injektív, amikor $U = \Ker F$.
\end{itemize}
F'-t az \textbf{F által indukált homomorfizmusnak} nevezzük.	% TODO magyar név helyes-e
\end{theorem}

A faktortérnek van egy hasznos tulajdonsága, amivel nem halmazrendszerként, hanem altérként lehet kezelni.
\begin{theorem}
Legyen V egy $\mathbb{K}$ feletti lineáris tér, $U \subset V$ egy altere, $W \subset V$ pedig az U komplemens altere (tehát $V = W \oplus U$). Ekkor a $\pi_U$ kanonikus leképezés leszűkítése W-re
\begin{equation*}
W \longrightarrow V / U, \; w \longrightarrow w + U
\end{equation*}
izomorfia, azaz $W \cong V / U$.
\end{theorem}

% TODO talán kivezető szöveg

% TODO esetleg 4.2.8 tétel ~ ennek analógja funkcionálokra (ha ezt később használjuk)

\section{Kompakt operátorok}
\label{sec:kompakt}

A Fredholm-operátorok konstrukciójához érdemes feleleveníteni a kompakt operátorok fogalmát és néhány, a gyakorlat szempontjából hasznos tulajdonságukat. A következők során $(X, \norm{.}_X)$ és $(Y, \norm{.}_Y)$ normált terek.
\begin{definition}
$A \, : \, X \rightarrow Y$ operátor \textbf{kompakt}, ha bármely $U \subset X$ korlátos halmaznak a képe prekompakt, azaz $\overline{A[U]} \subset Y$ kompakt, valamint ha A korlátos, akkor \textbf{teljesen folytonosnak} is nevezzük. \cite{funkanal}
\end{definition}
A továbbiakban az $X \rightarrow Y$ közötti kompakt és lineáris operátorok halmazát $K(X, Y)$-al, $X = Y$ esetén $K(X)$-szel jelöljük, ez zárt alteret alkot $L(X, Y)$-ban.
\begin{definition}
$A \in L(X, Y)$ operátor \textbf{véges rangú}, ha a képtere véges dimenziós ($\dim \Ima A < \infty$). A véges rangú operátorok halmazát $K_{fin}(X, Y)$-al rövidítjük.
\cite{funkanal,FcNotex}
\end{definition}
Belátható (\cite{funkanal}), hogy minden véges rangú operátor kompakt (és így a jelölés indokolt). \par
A kompakt operátorok bizonyos esetekben közelíthetőek véges rangú operátorokkal:
\begin{theorem}
Ha Y-ban van Schauder-bázis, akkor $A \in L(X, Y)$ pontosan akkor kompakt, ha határértéke véges rangú operátorok valamely sorozatának, azaz $K(X, Y) = \overline{K_{fin}(X, Y)}$. \cite{FcNotex}
\end{theorem}
Ezek a feltételek teljesülnek például szeparábilis Hilbert-terekben vagy $L^p$-terekben ($p \ge 1$). \par
Még tegyünk egy megállapítást a kompakt operátorok adjungáltjával kapcsolatban:
\begin{theorem}
Legyenek $(X, \norm{.}_X)$ és $(Y, \norm{.}_Y)$ Banach-terek, $A \in L(X, Y)$, ekkor
\begin{equation*}
A \in K(X, Y) \Leftrightarrow A^* \in K(Y^*, X^*),
\end{equation*}
azaz A pontosan akkor kompakt, ha $A^*$ adjungált operátora kompakt.
\cite{funkanal, lectures16and17}
\end{theorem}

\section{Fredholm-operátorok}
\label{sec:Fredholm}
% TODO Fredholm



%% 
 
%  Hivatkozások
\bibliography{hivatkozasok}	% TODO pontos adatok
\bibliographystyle{ieeetr}
 
\end{document}
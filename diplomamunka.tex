\documentclass[oneside, titlepage, 12pt, a4paper]{report}

\usepackage{amsmath}	% align
\usepackage{amsfonts}	% \mathbb
\usepackage{amssymb}	% \mathcal
\usepackage{mathtools}	% \norm definiálás
\usepackage[magyar]{babel}	% tartalomjegyzék felirat
\usepackage[utf8]{inputenc}	% magyar karakterek
\usepackage[T1]{fontenc}
\usepackage{graphicx}	% ábrák
\usepackage{setspace}	% sorköz \onehalfspacing

\newtheorem{theorem}{Tétel}[section]
\newtheorem{lemma}{Lemma}[section]
\newtheorem{definition}{Definíció}[section]
\newtheorem{statement}{Állítás}[section]

\DeclareMathOperator{\Ima}{Im}	% képtér operátor
\DeclareMathOperator{\Ker}{Ker}	% magtér operátor
\DeclareMathOperator{\coker}{coker}	% magtér operátor
\DeclareMathOperator{\codim}{codim}	% kodimenzió operátor
\DeclareMathOperator{\ind}{ind}	% index operátor

\DeclarePairedDelimiter\norm{\lVert}{\rVert}	% norma

\textwidth=6.truein \textheight=9.truein
\hoffset=-.5truein
\voffset=-.8truein

\begin{document}
\begin{titlepage}
% belső fedőlap

% minta forrása: https://github.com/shdnx/ELTE-LaTeX-Thesis-Base
\begin{minipage}{0.40\linewidth}
\includegraphics[scale=0.8]{./abrak/elte_logo_szines.jpg}
\end{minipage}
\begin{minipage}{0.50\linewidth}
\begin{center}
Eötvös Loránd Tudományegyetem \\
Informatikai Kar \\
Numerikus Analízis Tanszék
\end{center}
\end{minipage}

\hrule
\vfill

\begin{center}
\Huge
\textbf{A Ljapunov-Schmidt-módszer}
\normalsize
\end{center}

\vfill

\begin{minipage}[t]{0.5\linewidth}
\begin{flushleft}
\textbf{Dr. Kovács Sándor} \\
Adjunktus
\end{flushleft}
\end{minipage}
\begin{minipage}[t]{0.5\linewidth}
\begin{flushright}
\textbf{Lipták Bence Gábor} \\
Programtervező Informatikus MSc
\end{flushright}
\end{minipage}

\vfill

\begin{center}
Budapest, 2018.
\end{center}

\end{titlepage}

\tableofcontents

%% Bevezetés

\onehalfspacing
\chapter{Bevezetés}
\label{chap:Introduction}

% TODO bevezetés szövege

%% Alapozás: faktorterek, Fredholm operátorok

\onehalfspacing
\chapter{Funkcionálanalízis kiegészítés}	% TODOtalán cím
\label{chap:Funcanal_ext}

Ahhoz, hogy a Ljapunov-Schmidt-módszert ismertethessük szükségünk van a Fredholm-operátorok fogalmára, amihez elengedhetetlenek a faktorterek és a kompakt operátorok. A módszerhez emelett még az implicit függvény tétel (Banach-terekben) is szükséges.

% TODOtalán jelölések szekció: Lin op, korl Lin op halmazai

%
\section{Faktorterek}
\label{sec:Faktorter}

Először is ismertessük a faktorterek definícióját, és az alkalmazásunk szempontjából fontos tulajdonságait, a \cite{faktorter} könyv 4.2 fejezete alapján, ahol a bizonyítások is megtalálhatóak.
\begin{definition}[Faktortér]
Legyen V egy $\mathbb{K}$ feletti lineáris tér, $U \subset V$ pedig egy altere. A V tér U szerinti \textbf{faktortere} vagy \textbf{hányadostere}
\begin{equation}
V / U := \{v + U \mid v \in V\},
\end{equation}
ahol
\begin{equation}
v + U := \{v + u \mid u \in U \}.
\end{equation}
\end{definition}

Így egy lineáris térben egy altér segítségével definiáltunk egy halmazrendszert. A következő állítással megfogalmazzuk, hogy a kapott halmazok és az U altér között mi az összefüggés.
\begin{statement}
Ha V egy $\mathbb{K}$ feletti lineáris tér, $U \subset V$ egy altere, akkor $\forall v, v' \in V$-re
\begin{equation}
v + U = v' + U \qquad \Leftrightarrow \qquad v - v' \in U.
\end{equation}
\end{statement}

Ennek segítségével belátható, hogy ha  ''$v - v' \in U$'' feltétellel definiálunk egy relációt $V$ elemein, akkor az egy ekvivalenciareláció és az ekvivalenciaosztályok pedig a $V / U$ faktortér elemei. Ezután definiáljunk műveleteket a faktortéren.
\begin{theorem}
Legyen V egy $\mathbb{K}$ feletti lineáris tér, $U \subset V$ egy altere, ekkor
\begin{itemize}
\item
Megadunk	% TODOtalán átfogalmazás
\begin{align*}
V / U \times V / U & \longrightarrow V / U \\
\mathbb{K} \times V / U & \longrightarrow V / U
\end{align*}
leképezéseket a következő módon
\begin{align}
(v_1 + U) + (v_2 + U) &:= (v_1 + v_2) + U \\
\alpha (v + U) &:= \alpha v + U
\end{align}
melyek jól definiáltak.
\item
A $V / U$ faktortér ezekkel a műveletekkel egy $\mathbb{K}$ feletti lineáris tér.
\item
$\pi_U$ az úgynevezett \textbf{kanonikus leképezés}, melyet
\begin{equation*}
\pi_U(v) := v + U \qquad (v \in V)
\end{equation*}
módon definiálunk lineáris operátor $V$ és $V / U$ között.
\end{itemize}
\end{theorem}

A faktortér konstrukciója nagyon hasonlít az egész számok körében létesített maradékosztályokra, és rögzített $U \subset V$ esetén ugyanaz a jelölés is alkalmazható:
\begin{equation*}
\overline{v} := v + U \qquad (v \in V),
\end{equation*}
ezzel a jelöléssel a műveletek:
\begin{align*}
\overline{v} + \overline{w} &= \overline{v + w} &(v, w \in V) \\
\alpha \overline{v} &= \overline{\alpha v} &(\alpha \in \mathbb{K}, v \in V).
\end{align*}

Még fontos észrevétel, hogy a $\pi_U$ leképezés magtere pontosan az $U$ halmaz, valamint az operátor szürjektív is, amiből kapunk a faktortér dimenziójára egy összefüggést:
\begin{theorem}
Legyen V egy $\mathbb{K}$ feletti lineáris tér, $U \subset V$ egy altere, ekkor
\begin{equation}
\dim V / U + \dim U = \dim V.
\end{equation}
\end{theorem}

Ha egy $v \in V$ elemet egy $u \in U$ elemmel eltolunk ($U \subset V$ altér), akkor a $V / U$ faktortérbeli $\pi_U$ általi képe változatlan, ez a konstrukció alkalmas arra, hogy olyan függvényeket vizsgáljunk, amiknek az értéke az U altéren konstans, speciális esetben 0. Az ilyen leképezésekről szól a következő tétel:	% TODOtalán átfogalmazni
\begin{theorem}[Homomorfiatétel vektorterekre]
Legyen V egy $\mathbb{K}$ feletti lineáris tér, $F : V \rightarrow W$ egy lineáris leképezés, $U \subset V$ egy altér amire $U \subset \Ker F$. Ekkor egyértelműen létezik $F' : V / U \rightarrow W$ amivel $F = F' \circ \pi_U$. Emellett
\begin{itemize}
\item
$\Ima F = \Ima F'$, illetve F' pontosan akkor szürjektív, amikor F is,
\item
$\Ker F' = (\Ker F) / U$, illetve F' pontosan akkor injektív, amikor $U = \Ker F$.
\end{itemize}
F'-t az \textbf{F által indukált homomorfizmusnak} nevezzük.	% TODOmagyar név helyes-e
\end{theorem}

A faktortérnek van egy hasznos tulajdonsága, amivel nem halmazrendszerként, hanem altérként lehet kezelni.
\begin{theorem}
Legyen V egy $\mathbb{K}$ feletti lineáris tér, $U \subset V$ egy altere, $W \subset V$ pedig az U komplemens altere (tehát $V = W \oplus U$). Ekkor a $\pi_U$ kanonikus leképezés leszűkítése W-re
\begin{equation*}
\pi_{\mkern 1mu \vrule height 2ex\mkern2mu U} : W \longrightarrow V / U, \; \pi_{\mkern 1mu \vrule height 2ex\mkern2mu U}(w) = w + U
\end{equation*}
izomorfia, azaz $W \cong V / U$.
\end{theorem}

% TODOtalán kivezető szöveg

% TODOtalán 4.2.8 tétel ~ ennek analógja funkcionálokra (ha ezt később használjuk)

%
\section{Kompakt operátorok}
\label{sec:kompakt}

A Fredholm-operátorok konstrukciójához érdemes feleleveníteni a kompakt operátorok fogalmát és néhány, a gyakorlat szempontjából hasznos tulajdonságukat. A következők során $(X, \norm{.}_X)$ és $(Y, \norm{.}_Y)$ normált terek.
\begin{definition}[Kompakt operátor]
$A : X \rightarrow Y$ operátor \textbf{kompakt}, ha bármely $U \subset X$ korlátos halmaznak a képe prekompakt, azaz $\overline{A[U]} \subset Y$ kompakt, valamint ha A korlátos, akkor \textbf{teljesen folytonosnak} is nevezzük. \cite{funkanal}
\end{definition}

A továbbiakban az $X \rightarrow Y$ közötti kompakt és lineáris operátorok halmazát $K(X, Y)$-al, $X = Y$ esetén $K(X)$-szel jelöljük, ez zárt alteret alkot $L(X, Y)$-ban.
\begin{definition}
$A \in L(X, Y)$ operátor \textbf{véges rangú}, ha a képtere véges dimenziós ($\dim \Ima A < \infty$). A véges rangú operátorok halmazát $K_{fin}(X, Y)$-al rövidítjük.
\cite{funkanal,FcNotex}
\end{definition}
Belátható (\cite{funkanal}), hogy minden véges rangú operátor kompakt (és így a jelölés indokolt). \par

A kompakt operátorok bizonyos esetekben közelíthetőek véges rangú operátorokkal:
\begin{theorem}
Ha Y-ban van Schauder-bázis, akkor $A \in L(X, Y)$ pontosan akkor kompakt, ha határértéke véges rangú operátorok valamely sorozatának, azaz $K(X, Y) = \overline{K_{fin}(X, Y)}$. \cite{FcNotex}
\end{theorem}
Ezek a feltételek teljesülnek például szeparábilis Hilbert-terekben vagy $L^p$-terekben ($p \ge 1$). \par

Még tegyünk egy megállapítást a kompakt operátorok adjungáltjával kapcsolatban:
\begin{theorem}
Legyenek $(X, \norm{.}_X)$ és $(Y, \norm{.}_Y)$ Banach-terek, $A \in L(X, Y)$, ekkor
\begin{equation*}
A \in K(X, Y) \Leftrightarrow A^* \in K(Y^*, X^*),
\end{equation*}
azaz A pontosan akkor kompakt, ha $A^*$ adjungált operátora kompakt.
\cite{funkanal, lectures16and17}
\end{theorem}
% TODOtalán kivezető szöveg

%
\section{Fredholm-operátorok}
\label{sec:Fredholm}

Végül beszéljünk a Fredholm-operárokról és lehetőségekről az előállításukra. A továbbiakban $(X, \norm{.}_X)$ és $(Y, \norm{.}_Y)$ Banach-terek.
\begin{definition}[Fredholm-operátor]
$T \in L(X, Y)$ \textbf{Fredholm-operátor}, ha az alábbiak teljesülnek:
\begin{itemize}
\item
$\dim \Ker T < \infty$,
\item
T[X] zárt Y-ban,
\item
$\dim (Y / T[X]) < \infty$.
\end{itemize} % TODOmagyar nevek
Ekkor a T operátor \textbf{indexe} $\ind T := \dim \Ker T - \dim (Y / T[X]) \in \mathbb{Z}$, T \textbf{cokernele} $\coker T := Y / T[X]$ (azaz Y-ban a T képtere szerinti faktortér), valamint a cokernel dimenziója az operátor \textbf{kodimenziója} $\codim T := \dim \coker T$. \cite{faktorter, FcNotex, lectures16and17, diffun2}
\end{definition}
Az X és Y közötti Fredholm-operátorok halmazát a továbbiakban $\mathcal{F}(X, Y)$-al jelöljük. Belátható, hogy a második feltétel (T képterének a zártsága) a másik kettőből következik \cite{lectures16and17, diffun2}. \par

Most nézzük meg, hogy bizonyos függvény műveletek hatására változik-e a Fredholm-tulajdonság.	% TODOtalán átírni
\begin{theorem}
Legyenek $(X, \norm{.}_X)$, $(Y, \norm{.}_Y)$ és $(Z, \norm{.}_Z)$ Banach-terek, ekkor:
\begin{itemize}
\item
$A \in \mathcal{F}(X, Y)$ és $B \in \mathcal{F}(Y, Z)$, ekkor $B \circ A \in \mathcal{F}(X, Z)$ és $ \ind (B \circ A) = \ind B + \ind A$,
\item
$A \in \mathcal{F}(X, Y)$, ekkor $A^* \in \mathcal{F}(X^*, Y^*)$ és $\ind A^* = - \ind A$,
\item
$\mathcal{F}(X, Y)$ nyílt részhalmaza L(X, Y)-nak, és az $\ind : \mathcal{F}(X, Y) \rightarrow \mathbb{Z}$ függvény lokálisan konstans.
\end{itemize}
Tehát Fredholm-operátorok kompozíciója Fredholm-operátor, valamint Fredholm-operátor adjungáltja is Fredholm-operátor. \cite{FcNotex, diffun2}
\end{theorem}
% TODOtalán a lokálisan konstansságból következően a kis perturbáció tétel is?

Korábban említettük a kompakt operátorok és a Fredholm-operátorok kapcsolatát, erről szól a következő állítás.
\begin{theorem}
$T \in L(X, Y)$ bijektív, $K \in L(X, Y)$ kompakt, ekkor $T + K$ Fredholm-operátor és $\ind(T + K) = 0$. Ennek speciális esete amikor $(X, \norm{.}_X) = (Y, \norm{.}_Y)$ és T az identitás X-en. \cite{FcNotex,diffun2}
\end{theorem}

Amennyiben ilyen módon kaptunk egy Fredholm-operátort, akkor ahhoz (a kompakt operátorok altér-tulajdonságából kifolyólag) egy kompakt operátort hozzáadva is Fredholm-operátort kapunk, ez igaz tetszőleges konstrukció esetén is:
\begin{theorem}
$K \in K(X, Y)$ és $F \in \mathcal{F}(X, Y)$ esetén $K + F \in \mathcal{F}(X, Y)$, valamint $\ind(K + F) = \ind F$. \cite{FcNotex}
\end{theorem}

Mivel a Fredholm-operátoroknál nem feltétel, hogy a magterük csak a tér nullelemét tartalmazza, ezért általában nem invertálhatóak, de egy hasonló tulajdonságot megadhatunk:	% TODOtalán átírni
\begin{theorem}
$T \in L(X, Y)$ pontosan akkor Fredholm-operátor, ha létezik $B \in L(Y, X)$, $K_X \in K(X)$, $K_Y \in K(Y)$ úgy, hogy
\begin{equation*}
BT = I_{\mkern 1mu \vrule height 2ex\mkern2mu X} + K_X, TB = I_{\mkern 1mu \vrule height 2ex\mkern2mu Y} + K_Y.
\end{equation*}
Azaz a Fredholm-tulajdonság lényegében az invertálhatóság modulo kompakt operátort jelenti. \cite{FcNotex, diffun2}
\end{theorem}
% TODOtalán kivezető szöveg

%
\section{Implicit függvény tétel}
\label{sec:implicitfvtetel}

A módszer használata során az eredmény eléréséhez szükségünk lesz az implicit függvény tételre Banach-terekben, illetve ahhoz kötődően a Fréchet-derivált fogalmára. $(X, \norm{.}_X)$, $(Y, \norm{.}_Y)$ és $(Z, \norm{.}_Z)$ a következők során Banach-terek.

\begin{definition}
$F : X \times Y \rightarrow Z$ \textbf{Fréchet-differenciálható} X-ben az $(u_0, v_0)$ pontban, ha létezik $(D_xF)(u_0, v_0) \in L(X, Z)$ úgy, hogy
\begin{equation*}
\lim_{h \to 0} \frac{\norm{F(u_0 + h, v_0) - F(u_0, v_0) - (D_xF)(u_0, v_0)h}_Z}{\norm{h}_X} = 0.
\end{equation*}
\end{definition}
Látható, hogy egy ponthoz tartozó derivált (a valós, többdimenziós esethez hasonlóan) egy függvény.

\begin{theorem}[Implicit függvény tétel]
\label{implicit}
$F : X \times Y \rightarrow Z$ folytonos, $(u_0, v_0) \in X \times Y$, $F(u_0, v_0) = 0$, $(D_xF)(u_0, v_0)$ bijektív és folytonos. Ekkor létezik $(u_0, v_0)$-nak olyan $U \times V \subset X \times Y$ környezete és $G:V \rightarrow U$ függvény, amivel $G(v_0) = u_0$ és
\begin{equation*}
F(G(v), v) = 0 \qquad (\forall v \in V).
\end{equation*}
Ezen felül minden $U \times V$-beli megoldás ebben a formában áll elő. \cite{IFaLS}
\end{theorem}
% TODOtalán kivezető szöveg

%%

\onehalfspacing
\chapter{A Ljapunov-Schmidt-módszer}
\label{chap:modszer}

A Ljapunov-Schmidt-módszer célja, hogy egy végtelen-dimenziós feladatot egy jobban kezelhető, véges dimenziós feladatra redukáljunk. \par
Először is nézzük meg az általános esetet (\cite{8.6TLSM}). $(X, \norm{.}_X)$, $(Y, \norm{.}_Y)$ és $(\Lambda, \norm{.}_\Lambda)$ legyenek Banach-terek ($\Lambda$ a paramétertér szerepét tölti be), 0-val pedig értelemszerűen a megfelelő tér nullelemét jelöljük.\par
Legyen $F : X \rightarrow Z$ egy korlátos lineáris operátor, $N : X \times \Lambda \rightarrow Z$ pedig egy folytonosan (Fréchet-)differenciálható leképezés úgy, hogy
\begin{align*}
N(0, 0) &= 0, \\
(D_xN)(0,0) &= 0.
\end{align*}
Keressük $x \in X$-ben az alábbi egyenlet nemtriviális ($\lambda \ne 0$) megoldásait:
\begin{equation}
Fx = N(x, \lambda).
\end{equation}
Ha $D_xF(0)$ izomorfizmus $X$ és $Z$ között, akkor $A(x, \lambda) := Fx - N(x, \lambda)$ jelöléssel
\begin{equation*}
A(0, 0) = 0,
\end{equation*}
\begin{equation*}
D_xA(0, 0) = D_xF(0) \in L(X, Z) \, \text{folytonos bijekció},
\end{equation*}
így alkalmazható az implicit függvény tétel, ami megad egy $f : V \rightarrow U$ folytonos függvényt ($U \times V \subset X \times \Lambda$ a $(0, 0)$ egy nyílt környezete), amivel
\begin{equation*}
f(0) = 0,
\end{equation*}
\begin{equation*}
0 = A(f(\lambda), \lambda) = F(f(\lambda)) - N(f(\lambda), \lambda) \qquad (\lambda \in V),
\end{equation*}
és így 
\begin{equation*}
F(f(\lambda)) = N( f(\lambda), \lambda) \qquad (\lambda \in V),
\end{equation*}
ami megoldja az eredeti egyenletünket a $(0, 0)$ egy környezetében.\par
A Ljapunov-Schmidt-módszer bizonyos feltételek mellett megoldja a feladatot abban az esetben, ha $D_xF(0)$ nem izomorfizmus.	% TODO cont


% TODO

%% 
 
%  Hivatkozások
\bibliography{hivatkozasok}	% TODO pontos adatok
\bibliographystyle{ieeetr}
 
\end{document}
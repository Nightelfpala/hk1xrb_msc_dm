\documentclass[oneside, titlepage, 12pt, a4paper]{report}

\usepackage{amsmath}	% align
\usepackage{amsfonts}	% \mathbb
\usepackage[magyar]{babel}	% tartalomjegyzék felirat
\usepackage[utf8]{inputenc}	% magyar karakterek
\usepackage[T1]{fontenc}
\usepackage{graphicx}	% ábrák
\usepackage{setspace}	% sorköz \onehalfspacing

\newtheorem{theorem}{Tétel}[section]
\newtheorem{lemma}{Lemma}[section]
\newtheorem{definition}{Definíció}[section]
\newtheorem{statement}{Állítás}[section]

\textwidth=6.truein \textheight=9.truein
\hoffset=-.5truein
\voffset=-.8truein

\begin{document}
\begin{titlepage}
% belső fedőlap

% minta forrása: https://github.com/shdnx/ELTE-LaTeX-Thesis-Base
\begin{minipage}{0.40\linewidth}
\includegraphics[scale=0.8]{./abrak/elte_logo_szines.jpg}
\end{minipage}
\begin{minipage}{0.50\linewidth}
\begin{center}
Eötvös Loránd Tudományegyetem \\
Informatikai Kar \\
Numerikus Analízis Tanszék
\end{center}
\end{minipage}

\hrule
\vfill

\begin{center}
\Huge
\textbf{A Ljapunov-Schmidt-módszer}
\normalsize
\end{center}

\vfill

\begin{minipage}[t]{0.5\linewidth}
\begin{flushleft}
\textbf{Dr. Kovács Sándor} \\
Adjunktus
\end{flushleft}
\end{minipage}
\begin{minipage}[t]{0.5\linewidth}
\begin{flushright}
\textbf{Lipták Bence Gábor} \\
Programtervező Informatikus MSc
\end{flushright}
\end{minipage}

\vfill

\begin{center}
Budapest, 2018.
\end{center}

\end{titlepage}

\tableofcontents

%% Bevezetés

\onehalfspacing
\chapter{Bevezetés}
\label{chap:Introduction}

% TODO Introduction

%% Alapozás: faktorterek, Fredholm operátorok

\onehalfspacing
\chapter{Funkcionálanalízis kiegészítés}
\label{chap:Funcanal_ext}

Ahhoz, hogy a Ljapunov-Schmidt-módszert ismertethessük, szükségünk van a faktorterek és a Fredholm-operátorok fogalmaira.

\section{Faktorterek}
\label{sec:Faktorter}

Először is ismertessük a faktorterek definícióját, és az alkalmazásunk szempontjából fontos tulajdonságait, a \cite{faktorter} könyv 4.2 fejezete alapján.
\begin{definition}
Legyen V egy $\mathbb{K}$ feletti lineáris tér, $U \subset V$ pedig egy altere. A V tér U szerinti faktortere
\begin{equation}
V / U := \{v + U \mid v \in V\},
\end{equation}
ahol
\begin{equation}
v + U := \{v + u \mid u \in U \}.
\end{equation}
\end{definition}
Így egy lineáris térben egy altér segítségével definiáltunk egy halmazrendszert. A következő állítással megfogalmazzuk, hogy a kapott halmazok és az U altér között mi az összefüggés.
\begin{statement}
Ha V egy $\mathbb{K}$ feletti lineáris tér, $U \subset V$ egy altere, akkor $\forall v, v' \in V$-re
\begin{equation}
v + U = v' + U \Leftrightarrow v - v' \in U.
\end{equation}
\end{statement}
Ennek segítségével belátható, hogy ha  ''$v - v' \in U$'' feltétellel definiálunk egy relációt $V$ elemein, akkor az egy ekvivalenciareláció és az ekvivalenciaosztályok pedig a $V / U$ faktortér elemei. Ezután definiáljunk műveleteket a faktortéren.
\begin{theorem}
Legyen V egy $\mathbb{K}$ feletti lineáris tér, $U \subset V$ egy altere, ekkor
\begin{itemize}
\item
Megadunk	% TODO átfogalmazás
\begin{align*}
V / U \times V / U & \longrightarrow V / U \\
\mathbb{K} \times V / U & \longrightarrow V / U
\end{align*}
leképezéseket a következő módon
\begin{align}
(v_1 + U) + (v_2 + U) &:= (v_1 + v_2) + U \\
\alpha (v + U) &:= \alpha v + U
\end{align}
melyek jól definiáltak.
\item
A $V / U$ faktortér ezekkel a műveletekkel egy $\mathbb{K}$ feletti lineáris tér.
\item % TODO magyar név helyes-e
Az úgynevezett kanonikus $\pi_U$ leképezés, melyet
\begin{equation*}
\pi_U(v) := v + U \; (v \in V)
\end{equation*}
módon definiálunk lineáris operátor $V$ és $V / U$ között.
\end{itemize}
\end{theorem}

%% 
 
%  Hivatkozások
\bibliography{hivatkozasok}	% TODO
\bibliographystyle{ieeetr}
 
\end{document}